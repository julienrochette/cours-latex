% !TEX program = lualatex
\documentclass[11pt]{article}

% -------- LuaLaTeX : polices et langue --------
\usepackage{fontspec}
\setmainfont{Latin Modern Roman}
\setsansfont{Tex Gyre Heros}
%\renewcommand{\familydefault}{\sfdefault} % force le sans serif par défaut
\usepackage{polyglossia}
\setdefaultlanguage{french}

% -------- Mise en page --------
\usepackage[a4paper,margin=1cm]{geometry}
\usepackage{multicol}
\usepackage{fancyhdr}
\pagestyle{empty}
\usepackage[most]{tcolorbox}

% -------- Mathématiques --------
\usepackage{amsmath,amssymb,mathtools}
% \usepackage{siunitx}
% \sisetup{locale=FR}

\usepackage{enumitem}
\setlist[itemize]{left=0pt}
\setlist[enumerate]{left=0pt, label=\textbf{\alph*}.}

\usepackage{ProfCollege}
\usepackage{ProfMaquette}

\usepackage{tabularray}
%\usepackage{tabularx}

% -------- Divers --------
\newcommand{\ligne}{{\color{gray!60}\hrulefill}}

\setlength{\parindent}{0pt}

\begin{document}



\begin{Maquette}[IE]{
        Numero = 3, Code={}, Date = jeudi 16 octobre, Theme = Théorème de Thalès / Automatismes, Calculatrice = true
    }

    \begin{exercice}
        \brm{3}
        \begin{multicols}{2}
            \emph{La figure ci-contre n’est pas à l’échelle. Toutes les longueurs sont exprimées dans une même unité.}

            \begin{itemize}
                \item[] Déterminer si les triangles ABC et EFD sont semblables ou non.
                \item[] \ligne
                \item[] \ligne
                \item[] \ligne
                \item[] \ligne
                \item[] \ligne
                \item[] \ligne
                \item[] \ligne

            \end{itemize}


            \columnbreak

            \begin{center}
                \includegraphics[width = .85\linewidth]{Images/Évaluation 3 -semblables.png}
            \end{center}
        \end{multicols}
    \end{exercice}

    \vspace{.3cm}

    \begin{exercice}
        \brm{7}
        \raggedcolumns
        \begin{multicols}{2}
            \emph{La figure ci-contre est réalisée à main levée.}

            Les points A, E et D sont alignés, ainsi que les points B, E et C.

            Les droites (AB) et (CD) sont parallèles et on sait que :

            \begin{multicols}{2}
                \begin{itemize}
                    \item ED = \Lg{3,6}
                    \item EB = \Lg{7,2}
                    \item CD = \Lg{6}
                    \item AB = \Lg{9}
                \end{itemize}
            \end{multicols}

            \columnbreak

            \begin{center}
                \includegraphics[width=.6\linewidth]{Images/Évaluation 3 - thalès.png}
            \end{center}
        \end{multicols}
        \flushcolumns

        \begin{multicols}{2}
            \begin{enumerate}
                \item Déterminer la longueur EC.
                \item[] \ligne
                \item[] \ligne
                \item[] \ligne
                \item[] \ligne
                \item[] \ligne
                \item[] \ligne
                \item[] \ligne

                \item Démontrer que AD = \Lg{9}.
                \item[] \ligne
                \item[] \ligne
                \item[] \ligne
                \item[] \ligne
                \item[] \ligne
                \item[] \ligne
                \item[] \ligne
            \end{enumerate}
        \end{multicols}

    \end{exercice}

    \newpage

    Nom, prénom et classe : \ligne

    \vspace{.5cm}

    \begin{exercice}
        \brm{10}
        \begin{itemize}[itemsep=10pt]

            \item écriture réduite de $m \times (2 - 3m)$: \ligne
            \item écriture réduite de $5x+(-3+2x)$: \ligne
            \item écriture réduite de $-(10 - 9u) + 3u$: \ligne
            \item écriture réduite de $(a+3) \times (b+5)$: \ligne
            \item écriture réduite de $(x-2) \times (y-4)$: \ligne
            \item décomposition en facteurs premiers de 60 : \ligne
        \end{itemize}
        \begin{multicols}{2}
            \begin{itemize}[itemsep=10pt]
                \item étendue de $19\,\,;\,\,10\,\,;\,\,17$: \ligne
                \item médiane de $10 \,\,;\,\, 8 \,\,;\,\, 12 \,\,;\,\, 17$: \ligne
                \item moyenne de $10 \,\,;\,\,3\,\,;\,\,5$: \ligne
                \item médiane de $8 \,\,;\,\, 17 \,\,;\,\, 11 \,\,;\,\, 12 \,\,;\,\, 17$: \ligne
                \item nombre manquant dans $\dfrac{7}{2} = \dfrac{\ldots}{6}$: \ligne
                      \columnbreak
                \item $\dfrac{3}{4} + \dfrac{1}{2} =$ \ligne
                \item $3+\dfrac{2}{11}=$ \ligne

                \item écriture réduite de $\dfrac{10y}{2}$: \ligne
                \item écriture réduite de $\dfrac{8a}{12}$: \ligne
                \item écriture réduite de $\dfrac{5x}{20}$: \ligne

            \end{itemize}
        \end{multicols}

        \begin{tblr}{
            width = \textwidth,
            colspec = {|Q[l,8cm]|Q[c,2cm]|Q[c,2cm]|Q[c,2cm]|Q[c,2cm]|},
            hlines, vlines,
            row{1} = {font=\bfseries},
            cell{2-Z}{1-Z} = {valign=m},
            rowsep = 10pt,  % espace vertical dans toutes les cellules
            colsep = 5pt,
            }
            Question                                                                                                                                 & {\small Réponse A} & {\small Réponse B}     & {\small Réponse C}     & {\small Ta réponse}              \\
            Lors d’un agrandissement de rapport 3, les aires …                                                                                       & restent inchangées & sont multipliées par 3 & sont multipilées par 9 & \ligne                           \\
            On pioche un jeton au hasard \begin{center}
                                             \includegraphics[width=3cm]{Images/Évaluation 3 - proba 1.png}
                                         \end{center}  la probabilité d’obtenir un jeton bleu vaut …                                                                           & $5$                & $50 \%$                & $0,05$                 & \ligne \\
            On pioche un jeton au hasard \begin{center}
                                             \includegraphics[width=2cm]{Images/Évaluation 3 - proba 2.png}
                                         \end{center}  la probabilité d’obtenir un jeton bleu vaut …                                                                           & $0,6$              & $\dfrac{2}{5}$         & $55 \%$                & \ligne \\
            Dans un triangle ABC, on a $\widehat{\textrm{A}} = 60^\circ$ et $\widehat{\textrm{B}}=40^\circ$. L’angle $\widehat{\textrm{C}}$ mesure … & $40^\circ$         & $80^\circ$             & $90^\circ$             & \ligne                           \\
        \end{tblr}
    \end{exercice}
\end{Maquette}


\end{document}