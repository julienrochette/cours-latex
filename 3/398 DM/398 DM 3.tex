% !TEX program = lualatex
\documentclass[11pt]{article}

% -------- LuaLaTeX : polices et langue --------
\usepackage{fontspec}
\setmainfont{Latin Modern Roman}
\setsansfont{Tex Gyre Heros}
%\renewcommand{\familydefault}{\sfdefault} % force le sans serif par défaut
\usepackage{polyglossia}
\setdefaultlanguage{french}

% -------- Mise en page --------
\usepackage[a4paper,margin=1cm]{geometry}
\usepackage{multicol}
\usepackage{fancyhdr}
\pagestyle{empty}
\usepackage[most]{tcolorbox}

% -------- Mathématiques --------
\usepackage{amsmath,amssymb,mathtools}
% \usepackage{siunitx}
% \sisetup{locale=FR}

\usepackage{enumitem}
\setlist[itemize]{left=0pt}
\setlist[enumerate]{left=0pt, label=\textbf{\arabic*}.}

\usepackage{ProfCollege}
\usepackage{ProfMaquette}

%\usepackage{tabularray}
\usepackage{tabularx}

% -------- Divers --------
\newcommand{\ligne}{{\color{gray!60}\hrulefill}}

\setlength{\parindent}{0pt}

\begin{document}



\begin{Maquette}[DM]{
        Numero = 3, Code={}, Date = jeudi 4 décembre 2025, Niveau = Troisième
    }

 
    
    \begin{exercice}
        \brm{7}
        Voici le plan d’un cross (il n’est pas à l’échelle).

        \begin{center}
            \includegraphics[width=.5\linewidth]{Images/DM3-thales.png}
        \end{center}

        Les participants partent du point A et se rendent au point E en passant par les points B, C et D.

        C est le point d’intersection des droites (AE) et (BD).

        On sait que AC = \Lg[m]{400}, EC = \Lg[m]{1000} et AB = \Lg[m]{300}.

        \begin{enumerate}
            \item Calculer BC.
            \item Justifier pourquoi les droites (AB) et (ED) sont parallèles.
            \item Montrer que ED = \Lg[m]{750}.
            \item Déterminer la longueur réelle du parcours ABCDE.
        \end{enumerate}
    \end{exercice}
    
    \begin{exercice}
        \brm{8}
       Le diagramme en bâtons suivant représente la répartition des notes obtenues à un contrôle de mathématiques pour une classe de quatrième. Il n’y avait aucun élève absent le jour du contrôle.
       
       \begin{center}
        \includegraphics[width=.5\linewidth]{Images/DM3.png}
       \end{center}

       \begin{enumerate}
        \item Combien y a-t-il d’élèves dans cette classe de quatrième.
        \item Un élève préntend que la moyenne de la classe s’obtient en calculant
        
        \[ \dfrac{8+9+11+12+13+14+16}{7} \]

        Son professeur lui indique que c’est faux car il ne prend pas en compte les effectifs. Effectuer le bon calcul et donner la moyenne de la classe à ce contrôle. Arrondir la réponse au dixième d’unité.

        \item Calculer l’étendue de la série de notes de la classe.
        \item Déterminer la médiane de la série de notes de la classe.
        \item Calculer le pourcentage d’élèves de la classe ayant obtenu une note supérieure à 10.


       \end{enumerate}
    \end{exercice}


\end{Maquette}


\end{document}