% !TEX program = lualatex
\documentclass[xcolor={table,svgnames}]{beamer}

% Pour une meilleure écriture des mathématiques.
\usefonttheme[onlymath]{serif}


% -------- LuaLaTeX : polices et langue --------
\usepackage{fontspec}
\setmainfont{Latin Modern Roman}
\setsansfont{Tex Gyre Heros}
\renewcommand{\familydefault}{\sfdefault} % force le sans serif par défaut
\usepackage{polyglossia}
\setdefaultlanguage{french}

% ---------- Options générales ----------
\usetheme{Madrid}
\usecolortheme{orchid}
\usefonttheme{professionalfonts}

% Désactive footer et navbar
\setbeamertemplate{footline}{}
\setbeamertemplate{navigation symbols}{}

\usepackage{graphicx}
\usepackage{ProfCollege}

% ---------- Infos ----------
\title[Mon diaporama]{Questions flash}
%\subtitle{Classe de troisième}
%\author[Julien Rochette]{Julien Rochette \\ \texttt{julien-pierre-m.rochette@ac-dijon.fr}}
\date{\today}


% ---------- Début du document ----------
\begin{document}

\begin{frame}
  \titlepage
\end{frame}

\QFlash[Mesure]{\Lg{175}/%
\Lg[m]{}/%
\num{0.175}/%
\Lg[dm]{1}/%
\Lg[mm]{25}%
}

\QFlash[Simple]{%
Calcule la longueur AT (arrondis au centième d’unité)./
\begin{center}
  \Pythagore[FigureSeule]{ART}{6}{9}{}
\end{center}
}

\QFlash[Simple]{
  Quel résultat obtient-on si on applique le programme de calcul suivant avec le nombre $3$ ?/
  \ProgCalcul[%
Enonce,%
CouleurCadre=red,%
CouleurFond=pink!20,%
Nom=Programme de calcul,%
Largeur=8cm,%
Epaisseur=2pt,%
Pointilles=15mm%
]{%
Ajouter $2$,
Multiplier par $3$,
Soustraire le nombre de départ,
Élever au carré
}
}

\QFlash[Simple]{
On sait que les droites $(BC)$ et $(MN)$ sont parallèles.\\
De plus $AB=\Lg{15}$, $MN=\Lg{5}$, $AM=\Lg{6}$.\\
Déterminer la longueur $BC$./
\begin{center}
\Thales[FigureSeule]{ABCMN}{7}{AN}{35}{12}{AC}{BC}
\end{center}
}
\QFlash[Mesure]{\Aire{15}/%
\Aire[mm]{}/%
\num{0.15}/%
\Aire[dm]{2.5}/%
\Aire[mm]{25}%
}

\QFlash[Simple]{
Quelle figure a la plus grande aire ?/
\begin{itemize}
  \item un rectangle de longueur \Lg{7} et de largeur \Lg{5}
  \item un carré de côté \Lg{6}
  \item un disque de rayon \Lg{4}
\end{itemize}
}

\QFlash[Simple]{
  Dans chaque cas, donne l’écriture réduite …/
  \begin{itemize}
    \item $2x \times 3$
    \item $2x + 3$
    \item $x \times 2 + 3 \times x$
    \item $a \times 7 - 8 \times a$
    \item $b \times c \times b \times c \times b$
  \end{itemize}
}

\QFlash[Simple]{
  Calcule./
  \begin{itemize}
    \item $7-2$
    \item $5-10$
    \item $-2+5$
    \item $-3-2$
    \item $-3+4+6-4$
    \item $-1+4+5-2$
    \item $8-3+2-11+4$
  \end{itemize}
}


\QFlash[Simple]{%
Calcule la longueur AB./
\begin{center}
  \Pythagore[FigureSeule]{BAC}{97}{65}{}
\end{center}
}

\QFlash[Simple]{
  Calcule :/
  \begin{itemize}
    \item $\dfrac{5}{7}-\dfrac{2}{7}$
    \item $\dfrac{1}{2}+\dfrac{3}{4}$
    \item $\dfrac{3}{7}\times\dfrac{2}{11}$
    \item $\dfrac{5}{3}÷\dfrac{7}{2}$
  \end{itemize}
}

\end{document}
