% !TEX program = lualatex
\documentclass[11pt]{article}

% -------- LuaLaTeX : polices et langue --------
\usepackage{fontspec}
\setmainfont{Latin Modern Roman}
\setsansfont{Tex Gyre Heros}
%\renewcommand{\familydefault}{\sfdefault} % force le sans serif par défaut
\usepackage{polyglossia}
\setdefaultlanguage{french}

% -------- Mise en page --------
\usepackage[a4paper,margin=1cm]{geometry}
\usepackage{multicol}
\usepackage{fancyhdr}
\pagestyle{empty}
\usepackage[most]{tcolorbox}
\usepackage{graphicx}

% -------- Mathématiques --------
\usepackage{amsmath,amssymb,mathtools}
\usepackage{icomma}
% \sisetup{locale=FR}

\usepackage{enumitem}
\setlist[itemize]{left=0pt}
\setlist[enumerate]{left=0pt, label=\textbf{\alph*}.}

\usepackage{ProfCollege}
\usepackage{ProfMaquette}

\usepackage{tabularray}

% -------- Divers --------
\setlength{\parindent}{0pt}

\begin{document}

\begin{Maquette}[Fiche]{Theme=Triangles semblables et théorème de Thalès, Niveau=Troisième}

   \begin{multicols}{2}
      
      \begin{exercice}
      Ci-dessous, les triangles MER et LOA sont semblables.

         \includegraphics[width=.90\linewidth]{Images/ex1.png}
         Quel est l’homologue
         \begin{enumerate}
            \item du sommet L ?
            \item du sommet E ?
            \item du côté $[\textrm{ME}]$ ?
            \item de l’angle $\widehat{\textrm{LAO}}$ ?
         \end{enumerate}
      \end{exercice}

      \newcolumn

      \begin{exercice}
         \begin{itemize}
            \item Quel lien y a-t-il entre les mesures des trois angles d’un triangles ?
            \item Dans chaque cas, démontre que les deux triangles sont semblables.
         \end{itemize}
               
            \begin{center}
               \includegraphics[width
               =\linewidth]{Images/ex2.png}
            \end{center}
      \end{exercice}

         \begin{exercice}
            ART et ZEN sont deux triangles tels que :
            \begin{itemize}
               \item AR = \Lg{12}, AT = \Lg{14,4}, RT = \Lg{8,1}
               \item ZE = \Lg{10}, ZN = \Lg{5,4}, EN  = \Lg{8}
            \end{itemize}
            Réaliser deux figures à main levée.

            Ces deux triangles sont-ils semblables ? Justifier.
         \end{exercice}
   \end{multicols}
   
   \begin{multicols}{2}
      \begin{exercice}
         
         \includegraphics[width=\linewidth]{Images/ex4.png}
      \end{exercice}
\newcolumn
      \begin{exercice}
        Les deux triangles ABC et DEF de ce pendentif sont deux trianles isocèles semblables. Calculer la longueur AB.
        \begin{center}
        
         \includegraphics[width = .75\linewidth]{Images/ex5.png}
      \end{center}
      \end{exercice}
      \end{multicols}
      \newpage
   
   \begin{multicols}{2}
      \begin{exercice}
         Dans chaque cas, les droites en gras sont parallèles, écris les fractions de Thalès égales.

         \begin{center}
            \includegraphics[width=.85\linewidth]{Images/ex6.png}
         \end{center}
      \end{exercice}
   
      
      \begin{exercice}
         Sur la figure suivante, les droites (AB) et (TR) sont parallèles. De plus, on sait que SA = \Lg{4} ;  ST = \Lg{15} ; AB = \Lg{2,4} et SR = \Lg{7,5}.
         
         \begin{center}
            \includegraphics[width=.85\linewidth]{Images/ex7.png}
         \end{center}
         
         \begin{enumerate}
            \item Reporte les longueurs sur la figure.
            \item Calcule SB et RT.
         \end{enumerate}
      \end{exercice}
      
      \columnbreak
   \begin{exercice}
      Dans la situation suivante, les droites (MN) et (TS) sont parallèles.
      
      \begin{center}
         \includegraphics[width=.9\linewidth]{Images/ex8.png}
      \end{center}

      Calcule la longueur $y$ du segment[NT].
   \end{exercice}

   \begin{exercice}
      \begin{center}
         \includegraphics[width=\linewidth]{Images/ex9.png}
      \end{center}
   \end{exercice}
   \end{multicols}

   \begin{multicols}{2}
      \begin{exercice}
         Dans chaque situation, les deux droites en gras sont parallèles. Écrire les fractions de Thalès égales.

         \begin{center}
            \includegraphics[width=.8\linewidth]{Images/ex10.png}
         \end{center}

      \end{exercice}

      \begin{exercice}
         Dans chaque situation, les droites $(\textrm{d}_1)$ et $(\textrm{d}_2)$ sont parallèles. Relie chaque figure aux égalités de Thalès correspondantes.

         \begin{center}
            \includegraphics[width=\linewidth]{Images/ex11.png}
         \end{center}

      \end{exercice}
   \end{multicols}
   \begin{multicols}{2}
      
      \begin{exercice}
         Sur la figure ci-dessous, les droites (EF) et (HD) sont parallèles, et on sait de plus que GH = \Lg{15}, GF = \Lg{6}, GD = \Lg{14,2} et HD = \Lg{7,3}.
         \begin{center}
            \includegraphics[width=.95\linewidth]{Images/ex12.png}
         \end{center}
         Calculer les longueurs EF et EG.
      \end{exercice}

      \begin{exercice}
            Les points M, A, C sont alignés et les points N, A, B aussi. Les droites (MN) et (BC) sont parallèles. Calculer MN.

            \begin{center}
               \includegraphics[width=.9\linewidth]{Images/ex13.png}
            \end{center}
   
         \end{exercice}
   \end{multicols}
   \newpage

   \begin{exercice}
      
   \end{exercice}
\end{Maquette}

\end{document}
