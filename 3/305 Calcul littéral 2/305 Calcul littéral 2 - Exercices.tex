% !TEX program = lualatex
\documentclass[11pt]{article}

% -------- LuaLaTeX : polices et langue --------
\usepackage{fontspec}
\setmainfont{Latin Modern Roman}
\setsansfont{Tex Gyre Heros}
%\renewcommand{\familydefault}{\sfdefault} % force le sans serif par défaut
\usepackage{polyglossia}
\setdefaultlanguage{french}

% -------- Mise en page --------
\usepackage[a4paper,margin=1cm]{geometry}
\usepackage{multicol}
\usepackage{fancyhdr}
\pagestyle{empty}
\usepackage[most]{tcolorbox}

% -------- Mathématiques --------
\usepackage{amsmath,amssymb,mathtools}
\usepackage{icomma}
% \sisetup{locale=FR}

\usepackage{enumitem}
\setlist[itemize]{left=0pt}
\setlist[enumerate]{left=0pt, label=\textbf{\alph*}.}

\usepackage{ProfCollege}
\usepackage{ProfMaquette}

\usepackage{tabularray}

% -------- Divers --------
\setlength{\parindent}{0pt}
\newcommand{\ligne}{{\color{gray!60}\hrulefill}}

\begin{document}

\begin{multicols}{2}

    \begin{Maquette}[Fiche]{Theme=Calcul littéral 2, Niveau=Troisième}

        \begin{exercice}
            Réduis les quotients suivants :
            \begin{multicols}{2}

                \begin{enumerate}
                    \item $\dfrac{10x}{6}$
                    \item $\dfrac{15a}{3}$
                    \item $\dfrac{21b}{9}$
                    \item $\dfrac{5x}{5}$
                    \item $\dfrac{28y}{8}$
                    \item $\dfrac{20t}{10}$
                    \item $\dfrac{8x^2}{6}$
                    \item $\dfrac{-15c}{6}$
                    \item $\dfrac{20}{8x}$
                    \item $\dfrac{40a^2}{35a}$
                    \item $\dfrac{10ab}{8bc}$
                \end{enumerate}

            \end{multicols}

        \end{exercice}

        \begin{exercice}
            Réduis les quotients suivants :
            \begin{multicols}{2}

                \begin{enumerate}
                    \item $\dfrac{10x + 4}{2}$
                    \item $\dfrac{15 - 20a}{5}$
                    \item $\dfrac{3x+20}{3}$
                    \item $\dfrac{50 + 15t}{5}$
                    \item $\dfrac{5y - 28}{7}$
                    \item $\dfrac{18x - 2}{3}$
                    \item $\dfrac{55x^2 + 33t - 11}{11}$
                    \item $\dfrac{10x - 35}{-5}$
                    \item $\dfrac{-40+28a}{-4}$
                    \item $\dfrac{2}{8+6x}$
                \end{enumerate}
            \end{multicols}
        \end{exercice}
        \columnbreak
        \begin{exercice}

            Le nombre $3$ est-il solution des équations suivantes ?
            \begin{enumerate}
                \item $4x+2 = 15$
                \item[] \ligne
                \item[] \ligne
                \item $7-5x = -8$
                \item[] \ligne
                \item[] \ligne
                \item $4x-5 = 3x-1$
                \item[] \ligne
                \item[] \ligne
                \item[] \ligne
            \end{enumerate}

        \end{exercice}

        \begin{exercice}
            Récopier et résoudre les équations suivantes.
            \begin{enumerate}
                \item $x - 8 = 11$
                \item $8 x = 22$
                \item $7 + x = -5$
                \item $-4 x = 18 $
                \item $\dfrac{x}{4} = 3,5$
                \item $- x = 17$
            \end{enumerate}
        \end{exercice}

        \begin{exercice}
            Recopier et résoudre les équations suivantes.
            \begin{enumerate}
                \item $2x - 2 = 13$
                \item $3 z - 10 = 11$
                \item $37 = 5 x + 6$
                \item $-2 x + 6 = -10$
                \item $ 5\times (x - 8) = 52$
            \end{enumerate}
        \end{exercice}

        \newpage

        \begin{exercice}
            Traduire chacune des phrases par une équation, puis trouver la valeur manquante.

            \begin{enumerate}
                \item Le côté d’un triangle équilatéral mesure $x$ centimètres. Son périmètre est \Lg{18}.
                \item Un rectangle a une largeur qui mesure $\ell$ cm, et une longueur qui mesure \Lg{2} de plus que sa largeur. Son périmètre est \Lg{25}
                \item Un triangle ABC rectangle en A, d’aire \Aire{18}, est tel que AB = $x$ cm et AC = \Lg{6}.
                \item Igrek a $x$ ans. Dans 10 ans, il aura le triple de l’âge qu’il a maintenant.
                \item Zed a 40 pièces au total. $x$ pièces valent 1 €, les autres valent 2 €. Il possède 65 €.
            \end{enumerate}
        \end{exercice}

        \begin{exercice}

            \begin{enumerate}
                \item Dans une tirelire, il y a 19 billets uniquement de 5 € et de 10 €. La somme totale est de 130 €. Combien y a-t-il de billets de chaque sorte ?
                \item Un QCM contient 26 questions. Une bonne réponse rapporte 8 points et une mauvaise fait perdre 5 points. Une copie obtient la note 0, combien a-t-elle de bonnes réponses ?
                \item Trouver trois nombres entiers consécutifs dont la somme est 426.
                \item Trois personnes ont ensemble 110 ans. La deuxième a 15 ans de plus que la première. L’âge de l’aînée est égal à la somme des âges des deux autres. Trouver l’âge de ces trois personnes.
            \end{enumerate}

        \end{exercice}
        \columnbreak
        \begin{exercice}
            Léa et Tom affichent un même nombre sur leur calculatrice.
            \begin{itemize}
                \item Léa multiplie le nombre par 7, puis soustrait 5 au résultat obtenu.
                \item Tom multiplie le nombre par 3, puis ajoute 22 au résultat obtenu.
            \end{itemize}
            À la fin ils s’aperçoivent que leurs calculatrices affichent exactement le même résulat. Quel nombre commun avaient-ils affiché au départ ?
        \end{exercice}
        \begin{exercice}
            Résoudre les équations suivantes.
            \begin{enumerate}
                \item $3x + 28 = 9x + 7$
                \item $4x + 26 = 6x + 6$
                \item $7x + 11 = 3x + 6$
                \item $8x - 8 = 5x + 8$
                \item $4x - 10 = 2x + 3$
                \item $2x - 25 = 8x + 5$
            \end{enumerate}

        \end{exercice}

        \newpage

        \begin{exercice}
            Ixe et Igrek affichent un même nombre sur leur calculatrice.
            \begin{itemize}
                \item Ixe multiplie le nombre par 97, puis soustrait 51 au résultat obtenu.
                \item Igrek multiplie le nombre par 102, puis ajoute 74 au résultat obtenu.
            \end{itemize}
            À la fin ils s’aperçoivent que leurs calculatrices affichent exactement le même résulat. Quel nombre commun avaient-ils affiché au départ ?
        \end{exercice}
      
        \begin{exercice}
           \begin{enumerate}
            \item Partager 1500 € entre trois personnes de telle manière que la deuxième ait 150 € de plus que la  première et que la troisième ait 30 € de plus que la deuxième.
            \item Partager 2800 € entre trois personnes de manière que la première personne ait 350 € de plus que la deuxième et celle-ci 800 € de moins que la troisième.
           \end{enumerate}
        \end{exercice}

         \begin{exercice}
            Résoudre les cinq équations suivantes :
            \begin{enumerate}
                \item $12 - 5x - 2 = -4x + 2 - 5x$
                \item $3x - (4x - 8) = 2x + 3 - (x - 2)$
                \item $5(3 - x) - 4(2 - x) = 3(x + 4) - 6$
                \item $1 - (7 - 2x) - x = 5x - 2(x - 4)$
            \end{enumerate}
        \end{exercice}
    \end{Maquette}

\end{multicols}

\end{document}
