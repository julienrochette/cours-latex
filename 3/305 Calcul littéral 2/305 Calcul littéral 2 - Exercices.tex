% !TEX program = lualatex
\documentclass[11pt]{article}

% -------- LuaLaTeX : polices et langue --------
\usepackage{fontspec}
\setmainfont{Latin Modern Roman}
\setsansfont{Tex Gyre Heros}
%\renewcommand{\familydefault}{\sfdefault} % force le sans serif par défaut
\usepackage{polyglossia}
\setdefaultlanguage{french}

% -------- Mise en page --------
\usepackage[a4paper,margin=1cm]{geometry}
\usepackage{multicol}
\usepackage{fancyhdr}
\pagestyle{empty}
\usepackage[most]{tcolorbox}

% -------- Mathématiques --------
\usepackage{amsmath,amssymb,mathtools}
\usepackage{icomma}
% \sisetup{locale=FR}

\usepackage{enumitem}
\setlist[itemize]{left=0pt}
\setlist[enumerate]{left=0pt, label=\textbf{\alph*}.}

\usepackage{ProfCollege}
\usepackage{ProfMaquette}

\usepackage{tabularray}

% -------- Divers --------
\setlength{\parindent}{0pt}
\newcommand{\ligne}{{\color{gray!60}\hrulefill}}

\begin{document}

\begin{multicols}{2}

    \begin{Maquette}[Fiche]{Theme=Calcul littéral 2, Niveau=Troisième}

        \begin{exercice}
        Réduis les quotients suivants :
        \begin{multicols}{2}

            \begin{enumerate}
                \item $\dfrac{10x}{6}$
                \item $\dfrac{15a}{3}$
                \item $\dfrac{21b}{9}$
                \item $\dfrac{5x}{5}$
                \item $\dfrac{28y}{8}$
                \item $\dfrac{20t}{10}$
                \item $\dfrac{8x^2}{6}$
                \item $\dfrac{-15c}{6}$
                \item $\dfrac{20}{8x}$
                \item $\dfrac{40a^2}{35a}$
                \item $\dfrac{10ab}{8bc}$
            \end{enumerate}

        \end{multicols}

    \end{exercice}
       
    \begin{exercice}
        Réduis les quotients suivants :
        \begin{multicols}{2}
            
            \begin{enumerate}
                \item $\dfrac{10x + 4}{2}$
                \item $\dfrac{15 - 20a}{5}$
                \item $\dfrac{3x+20}{3}$
                \item $\dfrac{50 + 15t}{5}$
                \item $\dfrac{5y - 28}{7}$
                \item $\dfrac{18x - 2}{3}$
                \item $\dfrac{55x^2 + 33t - 11}{11}$
                \item $\dfrac{10x - 35}{-5}$
                \item $\dfrac{-40+28a}{-4}$
                \item $\dfrac{2}{8+6x}$
            \end{enumerate}
        \end{multicols}
        \end{exercice}
        \columnbreak
        \begin{exercice}
            
            Le nombre $3$ est-il solution des équations suivantes ?
            \begin{enumerate}
                \item $4x+2 = 15$
                \item[] \ligne
                \item[] \ligne
                \item $7-5x = -8$
                \item[] \ligne
                \item[] \ligne
                \item $4x-5 = 3x-1$
                \item[] \ligne
                \item[] \ligne
                \item[] \ligne
            \end{enumerate}

        \end{exercice}

        \begin{exercice}
            Récopier et résoudre les équations suivantes.
            \begin{enumerate}
                \item $x - 8 = 11$
                \item $8 x = 22$
                \item $7 + x = -5$
                \item $-4 x = 18 $
                \item $\dfrac{x}{4} = 3,5$
                \item $- x = 17$
            \end{enumerate}
        \end{exercice}

        \begin{exercice}
            Recopier et résoudre les équations suivantes.
            \begin{enumerate}
                \item $2x - 2 = 13$
                \item $3 z - 10 = 11$
                \item $37 = 5 x + 6$
                \item $-2 x + 6 = -10$
                \item $ 5\times (x - 8) = 52$
            \end{enumerate}
        \end{exercice}

    \end{Maquette}

\end{multicols}

\end{document}
