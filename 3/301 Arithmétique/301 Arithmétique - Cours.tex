% !TEX program = lualatex
\documentclass[11pt]{article}

% -------- LuaLaTeX : polices et langue --------
\usepackage{fontspec}
\setmainfont{Latin Modern Roman}
\setsansfont{Tex Gyre Heros}
%\renewcommand{\familydefault}{\sfdefault} % force le sans serif par défaut
\usepackage{polyglossia}
\setdefaultlanguage{french}

% -------- Mise en page --------
\usepackage[a4paper]{geometry}
\usepackage{multicol}
\usepackage{fancyhdr}
\pagestyle{empty}
\usepackage[most]{tcolorbox}

% -------- Mathématiques --------
\usepackage{amsmath,amssymb,mathtools}
% \usepackage{siunitx}
% \sisetup{locale=FR}

\usepackage{enumitem}
\usepackage{ProfCollege}

% -------- Divers --------
\setlength{\parindent}{0pt}

\begin{document}

\begin{tcolorbox}[colback=teal!10!white, colframe=teal!80!black]
\begin{center}
\large\textbf{Tâches de rentrée}
\end{center}
\end{tcolorbox}

\section{Rappels}

\subsection{Division euclidienne}

Effectuer la \emph{division euclidienne} d’un nombre entier $a$ par un nombre entier $b \neq 0$, c’est trouver deux nombres entier $q$ et $r$ tels que :
\[
a = b \times q + r \qquad \textrm{et} \qquad 0 \leq r < b
\]

\subsection{Divisibilité}

\section{Nombres premiers}
\subsection{Crible d’Ératosthène}

\section{Décomposition primaire}

\end{document}
