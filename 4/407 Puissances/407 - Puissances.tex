% !TEX program = lualatex
\documentclass[11pt]{article}

% -------- LuaLaTeX : polices et langue --------
\usepackage{fontspec}
\setmainfont{Latin Modern Roman}
\setsansfont{Tex Gyre Heros}
%\renewcommand{\familydefault}{\sfdefault} % force le sans serif par défaut
\usepackage{polyglossia}
\setdefaultlanguage{french}

% -------- Mise en page --------
\usepackage[a4paper,margin=1cm]{geometry}
\usepackage{multicol}
\usepackage{fancyhdr}
\pagestyle{empty}
\usepackage[most]{tcolorbox}

% -------- Mathématiques --------
\usepackage{amsmath,amssymb,mathtools}
\usepackage{icomma}
% \sisetup{locale=FR}

\usepackage{enumitem}
\setlist[itemize]{left=0pt}
\setlist[enumerate]{left=0pt, label=\textbf{\alph*}.}

\usepackage{ProfCollege}
\usepackage{ProfMaquette}

\usepackage{tabularray}

% -------- Divers --------
\newcommand{\ligne}{{\color{gray!60}\hrulefill}}
\setlength{\parindent}{0pt}

\begin{document}


\begin{Maquette}[Fiche]{Theme=Puissances, Niveau=Quatrième}

    \begin{multicols}{2}

        \begin{exercice}
            Voici une liste de mots :

            \begin{center}
                exposant --- puissance --- facteurs --- produit
            \end{center}

            Recopie chaque phrase en complétant les … à l’aide des mots précédents.

            \begin{enumerate}
                \item $3^7$ se lit « 3 … 7 ».
                \item $5^4$ est le … de quatre … tous égaux à 5.
                \item 8 est l’… de $6^8$.
                \item Le … de six … égaux s’écrit sous la forme d’une … d’… 6.
            \end{enumerate}
        \end{exercice}

        \begin{exercice}
            Écrire chaque puissance sous la forme du produit correspondant.
            \begin{multicols}{2}

                \begin{enumerate}

                    \item $3^4$
                    \item $5^7$
                    \item $2^3$
                    \item $1^8$
                    \item $2,5^3$
                    \item $13^1$
                    \item $10^4$
                    \item $7^5$
                    \item $(-4)^6$
                    \item $(-2)^3$
                \end{enumerate}
            \end{multicols}
        \end{exercice}

        \begin{exercice}
            Calculer de tête, ou avec la calculatrice si c’est nécessaire, et compléter.
            \begin{multicols}{2}

                \begin{enumerate}

                    \item $3^4 = \ldots$
                    \item $5^7 = \ldots$
                    \item $2^3 = \ldots$
                    \item $1^8 = \ldots$
                    \item $2,5^3 = \ldots$
                    \item $13^1 = \ldots$
                    \item $10^4 = \ldots$
                    \item $7^5 = \ldots$
                    \item $(-4)^6 = \ldots$
                    \item $(-2)^3 = \ldots$
                \end{enumerate}
            \end{multicols}
        \end{exercice}

        \begin{exercice}
            Léo pensait qu’une puissance d’un nombre négatif était toujours négative. À l’aide de sa calculatrice il a obtenu $(-6)^5 = -7776$ et $(-13)^4 = 28651$.
            \begin{enumerate}
                \item Quelle règle permet de savoir si une puissance d’un nombre négatif sera elle-même négative ?
                \item Sans calculer, entourer les nombres négatifs dans cette liste.
                \item \begin{multicols}{2}
                          \begin{itemize}
                              \item $(-3)^8$
                              \item $(-15)^9$
                              \item $(-9)^4$
                              \item $(-6)^7$
                              \item $(-88)^{11}$
                              \item $(-7)^1$
                              \item $(-9)^0$
                              \item $(-3)^{13}$
                              \item $(-12)^2$
                              \item $-3^4$

                          \end{itemize}
                      \end{multicols}
            \end{enumerate}
        \end{exercice}

    \end{multicols}
    \newpage

    \begin{multicols}{2}

        \begin{exercice}
            Lors d’un jeu télévisé les candidats doivent résoudre douze défis consécutifs. Avant de commencer, ils doivent choisir entre deux options pour les récompenses.
            \begin{itemize}
                \item Option A : chaque nouveau défi remporté fait gagner 600 €.
                \item Option B :
                      \begin{itemize}
                          \item le premier défi fait gagner 2 €
                          \item le deuxième défi fait gagner 4 €
                          \item le troisième défi fait gagner 8 €
                          \item et ainsi de suite : chaque nouveau défi remporté fait gagner deux fois plus que le défi précédent.
                      \end{itemize}
            \end{itemize}
            Si un candidat réussit les trois premiers défis et échoue au quatrième, alors il conserve les gains remportés lors des trois premiers défis mais le jeu s’arrête pour lui.

            Deux candidats, Tom et Léa, participent au jeu.
            \begin{enumerate}
                \item Tom pense qu’il réussira les cinq premiers défis et qu’il échouera au sixième. Quelle option doit-il choisir pour gagner la plus grosse récompense ?
                \item Léa pense qu’elle peut réussir les douze défis. Quelle option doit-elle cohisir pour gagner la plus grosse récompense ?
            \end{enumerate}
        \end{exercice}

        \begin{exercice}
            Donne l’écriture fractionnaire en calculant de tête. Vérifie ensuite tes calculs à l’aide d’une calculatrice, et donne une écriture décimale exacte (si possible) ou approchée au centième d’unité.
            \begin{multicols}{2}

                \begin{enumerate}
                    \item $2^{-3}$
                    \item $4^{-2}$
                    \item $7^{-1}$
                    \item $10^{-3}$
                    \item $2^{-5}$
                    \item $5^{-1}$
                    \item $3^{-4}$
                    \item $(-2)^{-2}$
                    \item $(-5)^{-3}$
                    \item $(-20)^{-2}$
                \end{enumerate}
            \end{multicols}
        \end{exercice}

        \begin{exercice}
            Détaille les calculs des puissances de 10 suivantes. Pour les puissances d’exposant négatif, donne les deux écritures (fractionnaire et décimale).
            \begin{enumerate}
                \item $10^3$
                \item $10^5$
                \item $10^1$
                \item $10^0$
                \item $10^{-2}$
                \item $10^{-4}$
            \end{enumerate}
        \end{exercice}
    \end{multicols}

    \newpage

    \begin{multicols}{2}
        \begin{exercice}
            Calculer.
            \begin{itemize}
                \item $A = 1 + 2^3 \times 4 - 5$
                \item $B = (1 + 2)^3 \times (4 - 5)$
                \item $C = 5 \times 4^3 \div 2 - 1$
                \item $D = (5 \times 4)^3 \div 2 - 1$
                \item $E = \dfrac{3^3-2^4}{2 \times 3^2 - 2^3 + 1}
            \end{itemize}
        \end{exercice}
        \columnbreak
        \begin{exercice}
            Calculer.

            \begin{itemize}
                \item $A = \left( 5 - \left( 4 - \left( 3 - \left( 2 - 1 \right)^1 \right) ^2 \right) ^3 \right) ^4$
                \item $B = \left( 5 - \left( 4 - \left( 3 - \left( 2 - 1 \right)^4 \right)^3 \right)^2 \right)^1$
                \item $C = \left( 1 - \left( 2 - \left( 3 - \left( 4 - 5 \right)^4 \right)^3 \right)^2 \right)^1$

            \end{itemize}
        \end{exercice}
    \end{multicols}
\end{Maquette}

\end{document}
