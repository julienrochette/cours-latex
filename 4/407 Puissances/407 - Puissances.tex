% !TEX program = lualatex
\documentclass[11pt]{article}

% -------- LuaLaTeX : polices et langue --------
\usepackage{fontspec}
\setmainfont{Latin Modern Roman}
\setsansfont{Tex Gyre Heros}
%\renewcommand{\familydefault}{\sfdefault} % force le sans serif par défaut
\usepackage{polyglossia}
\setdefaultlanguage{french}

% -------- Mise en page --------
\usepackage[a4paper,margin=1cm]{geometry}
\usepackage{multicol}
\usepackage{fancyhdr}
\pagestyle{empty}
\usepackage[most]{tcolorbox}

% -------- Mathématiques --------
\usepackage{amsmath,amssymb,mathtools}
\usepackage{icomma}
% \sisetup{locale=FR}

\usepackage{enumitem}
\setlist[itemize]{left=0pt}
\setlist[enumerate]{left=0pt, label=\textbf{\alph*}.}

\usepackage{ProfCollege}
\usepackage{ProfMaquette}

\usepackage{tabularray}

% -------- Divers --------
\newcommand{\ligne}{{\color{gray!60}\hrulefill}}
\setlength{\parindent}{0pt}

\begin{document}


\begin{Maquette}[Fiche]{Theme=Puissances, Niveau=Quatrième}

    \begin{multicols}{2}

        \begin{exercice}
            Voici une liste de mots :

            \begin{center}
                exposant --- puissance --- facteurs --- produit
            \end{center}

            Recopie chaque phrase en complétant les … à l’aide des mots précédents.

            \begin{enumerate}
                \item $3^7$ se lit « 3 … 7 ».
                \item $5^4$ est le … de quatre … tous égaux à 5.
                \item 8 est l’… de $6^8$.
                \item Le … de six … égaux s’écrit sous la forme d’une … d’… 6.
            \end{enumerate}
        \end{exercice}

        \begin{exercice}
            Écrire chaque puissance sous la forme du produit correspondant.
            \begin{multicols}{2}
                
                \begin{enumerate}
                    
                    \item $3^4$
                    \item $5^7$
                    \item $2^3$
                    \item $1^8$
                    \item $2,5^3$
                    \item $13^1$
                    \item $10^4$
                    \item $7^5$
                    \item $(-4)^6$
                    \item $(-2)^3$
                \end{enumerate}
            \end{multicols}
        \end{exercice}

         \begin{exercice}
            Calculer de tête, ou avec la calculatrice si c’est nécessaire, et compléter.
            \begin{multicols}{2}
                
                \begin{enumerate}
                    
                    \item $3^4 = \ldots$
                    \item $5^7 = \ldots$
                    \item $2^3 = \ldots$
                    \item $1^8 = \ldots$
                    \item $2,5^3 = \ldots$
                    \item $13^1 = \ldots$
                    \item $10^4 = \ldots$
                    \item $7^5 = \ldots$
                    \item $(-4)^6 = \ldots$
                    \item $(-2)^3 = \ldots$
                \end{enumerate}
            \end{multicols}
        \end{exercice}

        \begin{exercice}
            Léo pensait qu’une puissance d’un nombre négatif était toujours négative. À l’aide de sa calculatrice il a obtenu $(-6)^5 = -7776$ et $(-13)^4 = 28651$.
            \begin{enumerate}
                \item Quelle règle permet de savoir si une puissance d’un nombre négatif sera elle-même négative ?
                \item Sans calculer, entourer les nombres négatifs dans cette liste.
                \item \begin{multicols}{2}
                    \begin{itemize}
                        \item $(-3)^8$
                        \item $(-15)^9$
                        \item $(-9)^4$
                        \item $(-6)^7$
                        \item $(-88)^{11}$
                        \item $(-7)^1$
                        \item $(-9)^0$
                        \item $(-3)^{13}$
                        \item $(-12)^2$
                        \item $-3^4$
                        
                    \end{itemize}
                \end{multicols}
            \end{enumerate}
        \end{exercice}

    \end{multicols}
\end{Maquette}

\end{document}
