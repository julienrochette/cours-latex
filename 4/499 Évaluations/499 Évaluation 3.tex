% !TEX program = lualatex
\documentclass[11pt]{article}

% -------- LuaLaTeX : polices et langue --------
\usepackage{fontspec}
\setmainfont{Latin Modern Roman}
\setsansfont{Tex Gyre Heros}
%\renewcommand{\familydefault}{\sfdefault} % force le sans serif par défaut
\usepackage{polyglossia}
\setdefaultlanguage{french}

% -------- Mise en page --------
\usepackage[a4paper,margin=1cm]{geometry}
\usepackage{multicol}
\usepackage{fancyhdr}
\pagestyle{empty}
\usepackage[most]{tcolorbox}

% -------- Mathématiques --------
\usepackage{amsmath,amssymb,mathtools}
% \usepackage{siunitx}
% \sisetup{locale=FR}

\usepackage{enumitem}
\setlist[itemize]{left=0pt}
\setlist[enumerate]{left=0pt, label=\textbf{\alph*}.}

\newcommand{\checkbox}{\(\square\)}

\usepackage{ProfCollege}
\usepackage{ProfMaquette}

%\usepackage{tabularray}
\usepackage{tabularx}


% -------- Divers --------
\newcommand{\ligne}{{\color{gray!60}\hrulefill}}

\setlength{\parindent}{0pt}

\begin{document}



\begin{Maquette}[IE]{
        Numero = 3, Code={}, Date = jeudi 20 novembre, Theme = Nombres relatifs / Automatismes, Calculatrice = false
    }

    \begin{exercice}
        \brm{5} Effectue les calculs suivants.
        \begin{multicols}{2}
            \begin{enumerate}[itemsep=10pt]
                \item $ (-2) \times (-10) = $ \ligne
                \item $ 8 \times (-3) = $ \ligne
                \item $ (-3) \times (-2,5) = $ \ligne
                \item $ 9 \times (-1) = $ \ligne
                \item $ (-5) \times 5 = $ \ligne
                \item $ 100 \div (-2) = $ \ligne
                \item $ (-15) \div (-5) = $ \ligne
                \item $ \dfrac{17}{-1} = $ \ligne
                \item $ \dfrac{-66}{11} = $ \ligne
                \item $ - \dfrac{24}{-2} = $ \ligne
            \end{enumerate}

        \end{multicols}
    \end{exercice}

    \vspace{.5cm}
    \begin{exercice}
        \brm{5} Coche proprement les cases des calculs dont le résultat est un nombre positif.
        \begin{multicols}{2}
            \begin{itemize}[itemsep=10pt, label=\checkbox]
                \item $ -3+5 $
                \item $ -10 - 2 $
                \item $-5 + 20 - 2$
                \item $ 7 -(-2) $
                \item $ -3 - (-4)$
                \item $ (-5) \times 3$
                \item $ (-10) \times (-3) \times (-2)$
                \item $ (-1) \times 1 \times (-5) \times (-5) \times (-10) $
                \item $ (-1) \times 2 \times (-3) \times 4 \times (-5)$
                \item $\dfrac{(-5)\times 30 \times (-4)}{2 \times(-3) \times 5 }$
            \end{itemize}
        \end{multicols}
    \end{exercice}

    \vspace{.5cm}
    \begin{exercice}
        \brm{5}
        Effectue les deux calculs suivants. Rédige de manière détaillée.

        \begin{multicols}{2}

            \begin{itemize}[itemsep=10pt]
                \item $\textrm{A} = \left[(3 - 5) \times 4 + 2\right] \times 3 - 7$
                \item[] \ligne
                \item[] \ligne
                \item[] \ligne
                \item[] \ligne
                \item[] \ligne
                \item $\textrm{B} =3 - 5 \times \left[(4 + 2) \times 3 - 7 \right]$
                \item[] \ligne
                \item[] \ligne
                \item[] \ligne
                \item[] \ligne
                \item[] \ligne
            \end{itemize}

        \end{multicols}

    \end{exercice}

    \vspace{.5cm}
    \begin{exercice}
        \begin{multicols}{2}
            \begin{itemize}[itemsep=10pt]
                \brm{10}
                \item le carré de $4$ : \ligne
                \item $10$ au carré : \ligne
                \item $11^2$ : \ligne
                \item le carré de $9$ : \ligne
                \item le carré de $12$ : \ligne
                \item écriture réduite de $5 \times x \times 2 \times x$ : \ligne
                \item écriture réduite de $ 5 \times x - 2 $ : \ligne
                \item écriture réduite de $5 \times x - 2 \times x$ : \ligne
                \item écriture réduite de $ 3a \times 2b$ : \ligne
                \item écriture réduite de $ 3a + 5b -a$ : \ligne

                \item médiane de la série $ 13 \, ; \, 11 \, ; \, 12 $: \ligne
                \item médiane de la série $ 14 \, ; \, 7 \, ; \, 13 \, ; \, 8 \, ; \, 7$ : \ligne
                \item médiane de la série $ 10 \, ; \, 20 \, ; \, 8 \, ; \, 15 \, ; 6$ : \ligne
                \item écriture décimale de $\dfrac{1}{4}$ : \ligne
                \item écriture décimale de $\dfrac{1}{2}$ : \ligne
                \item écriture décimale de $\dfrac{1}{10}$ : \ligne
                \item écriture décimale de $\dfrac{1}{2}$ : \ligne
                \item écriture décimale de $\dfrac{1}{5}$ : \ligne
                \item écriture réduite de $7x + (4 - 2x)$ : \ligne
                \item écriture réduite de $20 - (7x - 2)$ : \ligne
            \end{itemize}
        \end{multicols}
    \end{exercice}
\end{Maquette}


\end{document}