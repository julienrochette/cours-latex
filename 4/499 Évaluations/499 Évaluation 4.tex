% !TEX program = lualatex
\documentclass[11pt]{article} 

% -------- LuaLaTeX : polices et langue --------
\usepackage{fontspec}
\setmainfont{Latin Modern Roman}
\setsansfont{Tex Gyre Heros}
%\renewcommand{\familydefault}{\sfdefault} % force le sans serif par défaut
\usepackage{polyglossia}
\setdefaultlanguage{french}

% -------- Mise en page --------
\usepackage[a4paper,margin=1cm]{geometry}
\usepackage{multicol}
\usepackage{fancyhdr}
\pagestyle{empty}
\usepackage[most]{tcolorbox}

% -------- Mathématiques --------
\usepackage{amsmath,amssymb,mathtools}
% \usepackage{siunitx}
% \sisetup{locale=FR}

\usepackage{enumitem}
\setlist[itemize]{left=0pt}
\setlist[enumerate]{left=0pt, label=\textbf{\alph*}.}

\newcommand{\checkbox}{\(\square\)}

\usepackage{ProfCollege}
\usepackage{ProfMaquette}

%\usepackage{tabularray}
\usepackage{tabularx}
\usepackage{tabularray}


% -------- Divers --------
\newcommand{\ligne}{{\color{gray!60}\hrulefill}}

\setlength{\parindent}{0pt}

\begin{document}



\begin{Maquette}[IE]{
        Numero = 4, Code={}, Date = jeudi 18 décembre, Theme = Théorème de Pythagore / Puissances , Calculatrice = true
    }


    \begin{multicols}{2}

        \begin{exercice}
            \brm{3} Rédige les calculs nécessaires pour trouver la longueur manquante dans chaque triangle.

            \begin{enumerate}
                \item
                \item[] \begin{center}
                          \includegraphics[width=.6\linewidth]{Images/Évaluation 4 - Longueur Pythagore 1.png}
                      \end{center}

                \item
                \item[] \begin{center}
                          \includegraphics[width=.6\linewidth]{Images/Évaluation 4 - Longueur Pythagore 2.png}
                      \end{center}

            \end{enumerate}
        \end{exercice}
        \columnbreak

        \begin{exercice} \brm{2}
            Relie chaque triangle à l’égalité de Pythagore correspondante.

            \begin{tblr}{
                    colspec = {
                            Q[l,m,wd=2.5cm]
                            Q[l,m,wd=.7cm]
                            Q[l,m,wd=4cm]
                        },
                    rulesep = 0pt,
                    rowsep = 0pt,
                }
                \includegraphics[width=\linewidth]{Images/Évaluation 4 - Égalité Pythagore 1.png} & $\bullet$  \newline \newline \newline & $\bullet \quad \textrm{BC}^2 = \textrm{CA}^2 - \textrm{AB}^2 $ \newline \newline \newline \\
                \includegraphics[width=\linewidth]{Images/Évaluation 4 - Égalité Pythagore 2.png} & $\bullet$ \newline \newline \newline  & $\bullet \quad \textrm{AC}^2 = \textrm{BC}^2 - \textrm{BA}^2 $ \newline \newline \newline \\
                \includegraphics[width=\linewidth]{Images/Évaluation 4 - Égalité Pythagore 3.png} & $\bullet$  \newline \newline \newline & $\bullet \quad \textrm{AB}^2 = \textrm{CA}^2 + \textrm{CB}^2$ \newline \newline \newline  \\
            \end{tblr}

        \end{exercice}

    \end{multicols}

    \begin{exercice}
        \brm{4}
        Lana a acheté une étagère qu’elle souhaite placer contre un mur dans une pièce dont le plafond à une hauteur de \Lg[m]{2,40}. Voici un schéma de la situation, il n’est pas à l’échelle.

        \begin{center}
            \includegraphics[width=.8\linewidth]{Images/Évaluation 4 - Étagère.png}
        \end{center}

        Léa a assemblé l’étagère à plat sur le sol de la pièce ; elle est donc en position 1.

        Vérifie si elle pourra relever l’étagère pour atteindre la position 2 sans toucher le plafond.
    \end{exercice}

    \newpage

    \begin{multicols}{2}
        
        \begin{exercice}
            \brm{2} Recopie chaque phrase en complétant les pointillés.
            
            \begin{enumerate}
                \item $7^9$ se lit « 7 … 9 ».
            \item $5^4$ est le … de quatre … tous égaux à 5.
            \item 6 est l’… de $2^6$.
        \end{enumerate}
    \end{exercice}
    \columnbreak
    \begin{exercice}
        \brm{4}
        Sur une île, des chercheurs ont introduit une espèce de lapins qui se développe très rapidement : chaque mois la population totale est multipliée par 3.
        \begin{enumerate}
            \item Par combien la population est-elle multipliée durant une période de deux mois ? Et durant une préiode de cinq mois ? Donne les réponses sous forme de puissances.
            \item Sachant qu’il y avait initialement 10 lapins, quel était le nombre de lapins après un an ? (on suppose, de manière simpliste, qu’il n’y a eu aucun décès de lapin durant cette période)
        \end{enumerate}
    \end{exercice}
\end{multicols}


    \newpage

    Nom, prénom et classe : \ligne

    \vspace{.5cm}

    \begin{exercice}
        \brm{10}
        \begin{itemize}[itemsep=10pt]

            \item écriture réduite de $5 \times (3 + m)$: \ligne
            \item écriture réduite de $4 \times (2x - 3)$: \ligne
            \item écriture réduite de $b \times (5 + 2b)$: \ligne
        \end{itemize}
        \begin{multicols}{2}
            \begin{itemize}[itemsep=10pt]
                \item le carré de $12$ : \ligne
                \item la racine carrée de $100$ : \ligne
                \item $\sqrt{64}$ : \ligne
                \item $6^2$ : \ligne
                \item la racine carrée de $4$ : \ligne
                      \columnbreak
                \item étendue de $19\,\,;\,\,10\,\,;\,\,17$: \ligne
                \item médiane de $8 \,\,;\,\, 17 \,\,;\,\, 11 \,\,;\,\, 12 \,\,;\,\, 17$: \ligne
                \item médiane de $10 \,\,;\,\, 8 \,\,;\,\, 12 \,\,;\,\, 17$: \ligne
                \item moyenne de $10 \,\,;\,\,3\,\,;\,\,5$: \ligne
                \item moyenne de $8 \,\,;\,\,2\,\,;\,\,5\,\,;\,\,3$: \ligne
            \end{itemize}
        \end{multicols}

        \begin{tblr}{
            width = \textwidth,
            colspec = {|Q[l,8cm]|Q[c,2cm]|Q[c,2cm]|Q[c,2cm]|Q[c,2cm]|},
            hlines, vlines,
            row{1} = {font=\bfseries},
            cell{2-Z}{1-Z} = {valign=m},
            rowsep = 10pt,  % espace vertical dans toutes les cellules
            colsep = 5pt,
            }
            Question                                                       & {\small Réponse A} & {\small Réponse B} & {\small Réponse C} & {\small Ta réponse}              \\
            $3^3 = \ldots$                                                 & $9$                & $27$               & $6$                & \ligne                           \\
            $8^1 = \ldots$                                                 & $8$                & $1$                & $-8$               & \ligne                           \\
            Lequel de ces nombres est positif ?                            & $(-3)^7$           & $-3^4$             & $(-5)^2$           & \ligne                           \\
            $7x - (3 - 2x) = \ldots$                                       & $5x - 3$           & $9x - 3$           & $9x + 3$           & \ligne                           \\
            $5 + (-4x +2) = \ldots$                                        & $4x +7$            & $-4x+3$            & $-4x+7$            & \ligne                           \\
            On pioche un jeton au hasard \begin{center}
                                             \includegraphics[width=3cm]{Images/Évaluation 4 - proba 1.png}
                                         \end{center}  la probabilité d’obtenir un jeton bleu vaut … & $5$                & $50 \%$            & $0,05$             & \ligne \\
            Le nombre $\sqrt{40}$ est compris entre …                      & $4$ et $5$         & $6$ et $7$         & $10$ et $11$       & \ligne                           \\
        \end{tblr}
    \end{exercice}
\end{Maquette}


\end{document}