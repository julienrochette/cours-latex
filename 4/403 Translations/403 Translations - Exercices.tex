% !TEX program = lualatex
\documentclass[11pt]{article}

% -------- LuaLaTeX : polices et langue --------
\usepackage{fontspec}
\setmainfont{Latin Modern Roman}
\setsansfont{Tex Gyre Heros}
%\renewcommand{\familydefault}{\sfdefault} % force le sans serif par défaut
\usepackage{polyglossia}
\setdefaultlanguage{french}

% -------- Mise en page --------
\usepackage[a4paper,margin=1cm]{geometry}
\usepackage{multicol}
\usepackage{fancyhdr}
\pagestyle{empty}
\usepackage[most]{tcolorbox}

% -------- Mathématiques --------
\usepackage{amsmath,amssymb,mathtools}
\usepackage{icomma}
% \sisetup{locale=FR}

\usepackage{enumitem}
\setlist[itemize]{left=0pt}
\setlist[enumerate]{left=0pt, label=\textbf{\arabic*}.}

\usepackage{ProfCollege}
\usepackage{ProfMaquette}

\usepackage{tabularray}

% -------- Divers --------
\setlength{\parindent}{0pt}

\begin{document}

\begin{Maquette}[Fiche]{Theme=Translations, Niveau=Quatrième}

\begin{exercice}
    \begin{center}
    \includegraphics[width=.75\linewidth]{Images/exercice1.png}
    \end{center}
    Dans chaque situation, la lettre L noire a été transformée géométriquement en la lettre blanche.
    \begin{enumerate}
        \item Quelles transformations ont été utilisées dans les situations 1 et 2 ?
        \item Et dans les situations 3 et 4 ?
    \end{enumerate}
\end{exercice}

\newpage

\begin{exercice}
    On s’intéresse au personnage auquel le point $A$ appartient. Identifie chaque figure qui est l’image du personnage par une \emph{translation}. Colorie ces figures en vert et trace un \emph{vecteur} correspondant à chaque translation.\\
    Pour les autres figures, indique ce qui ne va pas.
    \begin{center}
        \includegraphics[width=.75\linewidth]{Images/exercice2.png}
    \end{center}
\end{exercice}

\begin{exercice}
    \begin{enumerate}
        \item Trace l’image du quadrilatère $ABCD$ par la translation qui transforme $E$ en $F$.
        \item Trace l’image de $ABCD$ par la translation de vecteur $\overrightarrow{GH}$.
    \end{enumerate}
    \begin{center}
        \includegraphics[width=.75\linewidth]{Images/exercice3.png}
    \end{center}
\end{exercice}

\newpage

\begin{multicols}{2}
    
\begin{exercice}
    \includegraphics[width=.95\linewidth]{Images/exercice4.png}

    \begin{enumerate}
        \item Colorie l’intérieur du triangle CGH en jaune.
        \item Quelle est l’image du triangle CGH par la translation qui transforme C en E ? Colorie son intérieur en bleu.
        \item Quelle est l’image du triangle CGH par la translation de vecteur $\overrightarrow{\textrm{BF}}$ ? Colorie son intérieur en vert.
        \item On s’intéresse au triangle EBA. Colorie son intérieur en rouge.
            \begin{enumerate}[label=\textbf{\alph*.}]
                \item Pourquoi ce triangle ne peut-il pas être l’image de CGH par une translation ?
                \item Donne une transformation géométrique qui transforme CGH en EBA.
            \end{enumerate}
    \end{enumerate}
\end{exercice}

\columnbreak

\begin{exercice}
    \includegraphics[width=\linewidth]{Images/exercice5.png}

    \begin{enumerate}
        \item On s’intéresse au triangle AFG.
        \begin{enumerate}[label=\textbf{\alph*.}]
            \item Quelle est son image par la translation qui transforme G en R ?
            \item Quelle est son image par la symétrie axiale d’axe (LN) ?
            \item Quelle est son image par la symétrie centrale de centre H ?
        \end{enumerate}
        \item On s’intéresse au rectangle WRTY. Quelle est son image par la translation qui transforme X en H ?
    \end{enumerate}
\end{exercice}

\begin{exercice}
    \includegraphics[width=\linewidth]{Images/exercice6.png}
\end{exercice}

\end{multicols}

\end{Maquette}

\end{document}
