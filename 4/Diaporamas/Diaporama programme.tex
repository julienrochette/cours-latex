% !TEX program = lualatex
\documentclass[11pt]{beamer}

% -------- LuaLaTeX : polices et langue --------
\usepackage{fontspec}
\setmainfont{Latin Modern Roman}
\setsansfont{Tex Gyre Heros}
\renewcommand{\familydefault}{\sfdefault} % force le sans serif par défaut
\usepackage{polyglossia}
\setdefaultlanguage{french}

% ---------- Options générales ----------
\usetheme{Madrid}
\usecolortheme{orchid}
\usefonttheme{professionalfonts}

% Désactive footer et navbar
\setbeamertemplate{footline}{}
\setbeamertemplate{navigation symbols}{}

% Images, maths, code
\usepackage{graphicx}
\usepackage{booktabs}
\usepackage{hyperref}
\usepackage{listings}
\lstset{basicstyle=\ttfamily\small,breaklines=true}

% ---------- Infos ----------
\title[Mon diaporama]{Programme de mathématiques}
\subtitle{Classe de quatrième}
%\author[Julien Rochette]{Julien Rochette \\ \texttt{julien-pierre-m.rochette@ac-dijon.fr}}
\date{\today}

% ---------- Commande pour créer une section avec colonne et image ----------
\newcommand{\maSection}[3]{%
  \section{#1}
  \begin{frame}{\thesection. \insertsection}
    \begin{columns}[T,onlytextwidth]
      \column{0.48\textwidth}
      \begin{itemize}
        #2
      \end{itemize}

      \column{0.48\textwidth}
      \centering
      \includegraphics[width=\linewidth]{Images/#3.png}
    \end{columns}
  \end{frame}
}

% ---------- Début du document ----------
\begin{document}

\begin{frame}
  \titlepage
\end{frame}

\begin{frame}{Thèmes}
  \tableofcontents
\end{frame}

% ---------- Sections ----------
\maSection{Nombres et calculs}{
  \item Nombres relatifs
  \item Écriture fractionnaire
  \item Arithmétique
  \item Puissances
  \item Calcul littéral
}{1}

\maSection{Organisation et gestion de données, fonctions}{
  \item Proportionnalité, pourcentages, évolutions
  \item Statistiques
  \item Probabilités
  \item Fonctions
}{2}

\maSection{Grandeurs et mesures}{
  \item Unités
  \item Grandeurs simples
  \item Grandeurs composées
  \item Formules
}{3}

\maSection{Espace et géométrie}{
  \item Théorème de Pythagore
  \item Théorème de Thalès
  \item Transformations
  \item Pyramides / Cônes
}{4}

\maSection{Algorithmique et programmation}{
  \item Scratch
  \item Tableur
}{5}

\end{document}
