% !TEX program = lualatex
\documentclass[11pt]{article}

% -------- LuaLaTeX : polices et langue --------
\usepackage{fontspec}
\setmainfont{Latin Modern Roman}
\setsansfont{Tex Gyre Heros}
%\renewcommand{\familydefault}{\sfdefault} % force le sans serif par défaut
\usepackage{polyglossia}
\setdefaultlanguage{french}

% -------- Mise en page --------
\usepackage[a4paper,margin=1cm]{geometry}
\usepackage{multicol}
\usepackage{fancyhdr}
\pagestyle{empty}
\usepackage[most]{tcolorbox}

% -------- Mathématiques --------
\usepackage{amsmath,amssymb,mathtools}
% \usepackage{siunitx}
% \sisetup{locale=FR}

\usepackage{enumitem}
\setlist[itemize]{left=0pt}
\setlist[enumerate]{left=0pt, label=\textbf{\arabic*}.}

\usepackage{ProfCollege}
\usepackage{ProfMaquette}

%\usepackage{tabularray}
\usepackage{tabularx}

% -------- Divers --------
\newcommand{\ligne}{{\color{gray!60}\hrulefill}}

\setlength{\parindent}{0pt}

\begin{document}

\begin{Maquette}[DM]{
        Numero = 2, Code={}, Date = lundi 7 décembre, Niveau = Quatrième
    }

    \begin{exercice}
        \brm{4}
         Le diagramme en bâtons suivant représente la répartition des notes obtenues à un contrôle de mathématiques pour une classe de quatrième. Il n’y avait aucun élève absent le jour du contrôle.
       
       \begin{center}
        \includegraphics[width=.45\linewidth]{Images/DM2.png}
       \end{center}

       \begin{enumerate}
        \item Combien y a-t-il d’élèves dans cette classe de quatrième.
        \item Un élève préntend que la moyenne de la classe s’obtient en calculant
        
        \[ \dfrac{8+9+11+12+13+14+16}{7} \]

        Son professeur lui indique que c’est faux car il ne prend pas en compte les effectifs. Effectuer le bon calcul et donner la moyenne de la classe à ce contrôle. Arrondir la réponse au centième d’unité.
        \item Y a-t-il eu plus ou moins de la moitié des élèves qui ont obtenu une note supérieure ou égale à 13 ?
       \end{enumerate}
    \end{exercice}

    \begin{exercice}
        \brm{6}
        \begin{multicols}{2}
            Léa veut fabriquer un chapeau en forme de \emph{cône} pour se déguiser en sorcière lors de la fête d’Halloween.

            Voici une représentation en perspective cavalière de ce chapeau.

            Le rayon OM de la base mesure \Lg{9} et la hauteur OS mesure \Lg{30}

            \begin{center}
                \includegraphics[width=.5\linewidth]{Images/DM2-ex2-figure1.png}
            \end{center}
            
            \begin{enumerate}
                \item Démontrer que la longueur MS, arrondie au dixième, est environ égale à \Lg{31,3}.
                \item Léa souhaite vérifier que le chapeau sera adapté à son tour de tête qui mesure \Lg{56}. Parmi les trois grandeurs suivantes, indique celle que doit calculer Léa pour effectuer sa vérification :
                \begin{itemize}
                    \item l’aire de la base du cône
                    \item le périmètre de la base du cône
                    \item le volume du cône
                \end{itemize}
                
                \item Effectue le calcul pour vérifier si le chapeau sera adapté ou non.
            \end{enumerate}

            \begin{center}
                \includegraphics[width=.75\linewidth]{Images/DM2-ex2-figure2.jpg}

                Schéma illustrant la mesure du tour de tête.
            \end{center}
        \end{multicols}
    \end{exercice}
\end{Maquette}


\end{document}