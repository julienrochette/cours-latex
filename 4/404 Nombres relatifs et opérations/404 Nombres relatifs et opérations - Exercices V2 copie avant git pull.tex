% !TEX program = lualatex
\documentclass[11pt]{article}

% -------- LuaLaTeX : polices et langue --------
\usepackage{fontspec}
\setmainfont{Latin Modern Roman}
\setsansfont{Tex Gyre Heros}
%\renewcommand{\familydefault}{\sfdefault} % force le sans serif par défaut
\usepackage{polyglossia}
\setdefaultlanguage{french}

% -------- Mise en page --------
\usepackage[a4paper,margin=1cm]{geometry}
\usepackage{multicol}
\usepackage{fancyhdr}
\pagestyle{empty}
\usepackage[most]{tcolorbox}

% -------- Mathématiques --------
\usepackage{amsmath,amssymb,mathtools}
\usepackage{icomma}
% \sisetup{locale=FR}

\usepackage{enumitem}
\setlist[itemize]{left=0pt}
\setlist[enumerate]{left=0pt, label=\textbf{\alph*}.}

\usepackage{ProfCollege}
\usepackage{ProfMaquette}

\usepackage{tabularray}

% -------- Divers --------
\setlength{\parindent}{0pt}

\begin{document}


\begin{Maquette}[Fiche]{Theme=Nombres relatifs et opérations, Niveau=Quatrième}

    \begin{multicols}{2}
        \begin{exercice}
            Recopie et effectue les calculs suivants.
            \begin{enumerate}
                \item $7-10$
                \item $-3+5$
                \item $15-2$
                \item $-9-1$
                \item $-10+2$
                \item $(-3)+(-2)$
                \item $5+(-2)$
                \item $7-(-3)$
                \item $-10-(-2)$
            \end{enumerate}
        \end{exercice}

        \begin{exercice}
            Complète la pyramdie d’additions suivantes.\\
            (Le contenu d’une case doit toujours être égal au résultat de l’addition des deux cases situées en immédiatement en dessous.)
            \begin{center}

                \PyramideNombre[Largeur=1.5cm]{%
                    10,-4,3,-7,9,%
                    ~,~,~,~,%
                    ~,~,~,%
                    ~,~,%
                    ~}
            \end{center}
        \end{exercice}

        \begin{exercice}
            \begin{enumerate}
                \item Lucie prétend qu’elle est capable d’écrire un calcul qui a le même résultat que $(-5)\times 4$ en utilisant uniquement des additions. Comment fait-elle ? Quelle est donc le résultat du produit $(-5) \times 4$ ?
                \item Calcule de la même manière $(-10) \times 3$ et $3 \times (-2)$.
                \item Donne une règle qui permet, d’après toi, de multiplier deux nombres relatifs de signes contraires.
            \end{enumerate}
        \end{exercice}

        \begin{exercice}
            Recopie et effectue les calculs suivants.
            \begin{multicols}{2}

                \begin{enumerate}
                    \item $(-6) \times 10$
                    \item $7 \times (-2)$
                    \item $(-1) \times 8$
                    \item $(-2) \times 1,5$
                    \item $2,3 \times (-100)$
                    \item $1000 \times (-3,7)$
                \end{enumerate}
            \end{multicols}
        \end{exercice}

        \begin{exercice}
            Un jeu de cartes est composé de cartes avec des nombres relatifs qui font gagner (nombres positifs) et perdre (nombres négatifs) des points.
            Un joueur a des cartes dans sa main et son score total vaut $30$.
            Que se passe-t-il s’il …
            \begin{enumerate}
                \item … ajoute deux cartes $1$ dans sa main ?
                \item … ajoute deux cartes $-4$ dans sa main ?
                \item … enlève deux cartes $3$ de sa main ?
                \item … enlève deux cartes $-5$ de sa main ?
            \end{enumerate}
        \end{exercice}


    \end{multicols}
    \vspace{1cm}
    \begin{multicols}{2}
        \begin{exercice}
            Recopie et effectue les calculs suivants.
            \begin{multicols}{2}

                \begin{enumerate}
                    \item $(-6) \times (-2)$
                    \item $-7 \times (-10)$
                    \item $(-3) \times 5$
                    \item $(-2) \times 5$
                    \item $(-1,5) \times (-100)$
                    \item $1000 \times 3,7$
                    \item $-1,1 \times (-13)$
                    \item $0,02 \times (-1,4)$
                \end{enumerate}
            \end{multicols}
        \end{exercice}

        \columnbreak

        \begin{exercice}
            On s’intéresse aux quatre calculs suivants :
            \begin{enumerate}
                \item $(-2) \times (-3) \times (-5)$
                \item $(-2) \times (-5) \times (-7) \times (-11)$
                \item $(-2) \times 3 \times (-3) \times (-10)$
                \item $2 \times (-5) \times (-2) \times 5$
            \end{enumerate}
            Léa dit qu’il est facile de prédire le signe du résultat. Pourquoi a-t-elle raison ? Recopie chaque calcul et indique son résultat.
        \end{exercice}
    \end{multicols}
    \newpage
    \begin{exercice}
        Bla 8
    \end{exercice}
    \begin{exercice}
        Série 1 : A = –16, B = 14, C = 42, D11 = 40
        
        Série 2 : A = –18, B = 8, C = –52, D = –25
        
        Exercice suivant : A = 3, B = 7, C = –50, D = –20, E = –31, F = 2, 
        
        Expressions pour a = –2 et b = 3 : A = –21, B = 11, C = 16, D = 1, E = –16, F = 0
    \end{exercice}
    \vspace{1cm}
    \begin{exercice}
        Donne l’écriture décimale de chaque quotient :
        \begin{multicols}{5}
            $\textrm{A} = \dfrac{-20}{2}$\vspace{4mm}

            $\textrm{F} = \dfrac{-13}{-1}$\vspace{4mm}

            $\textrm{B} = \dfrac{-40}{-10}$\vspace{4mm}

            $\textrm{G} = -\dfrac{10}{5}$\vspace{4mm}

            $\textrm{C} = \dfrac{15}{-3}$\vspace{4mm}

            $\textrm{H} = -\dfrac{-6}{2}$\vspace{4mm}

            $\textrm{D} = \dfrac{-7}{-2}$\vspace{4mm}

            $\textrm{I} = -\dfrac{100}{-4}$\vspace{4mm}

            $\textrm{E} = \dfrac{-8}{8}$\vspace{4mm}

            $\textrm{I} = -\dfrac{-60}{-3}$\vspace{4mm}
        \end{multicols}
    \end{exercice}

    \begin{multicols}{2}

        \begin{exercice}
            Recopie et effectue chaque calcul.

            \textbf{Série 1}
            \begin{itemize}
                \item $A = 10 - 3 \times 4 - 7 \times 2$
                \item $B = (10 - 3) \times 4 - 7 \times 2$
                \item $C = \left[ (10 - 3) \times 4 - 7 \right] \times 2$
                \item $D = 10 - 3 \times (4 - 7 \times 2)$
            \end{itemize}

            \textbf{Série 2}
            \begin{itemize}
                \item $A = 3 - 5 \times 4 + 2 \times 3 - 7$
                \item $B = (3 - 5) \times \left[4 + 2 \times (3 - 7)\right]$
                \item $C = 3 - 5 \times \left[(4 + 2) \times 3 - 7 \right]$
                \item $D = \left[ (3 - 5) \times 4 + 2 \right] \times 3 - 7$
            \end{itemize}
        \end{exercice}
        \newcolumn
        \begin{exercice}
            Recopie et effectue chaque calcul.
            \begin{itemize}
                \item $A = \left[(-3) \times 7 + 6\right] ÷ (-5)$
                \item $B = 18 ÷ (-6) - 5 \times (-2)$
                \item $C = \left( 7 - (-3) \times 4\right) \times (-2) + (-12)$
                \item $D = 3 \times (-5) + (-25) ÷ 5$
                \item $E = 8 \times (-5) + 3 - (-48) ÷ 8$
                \item $F = (-4 \times 5 - 2) ÷ \left(2 \times (-6) + 1\right)$
            \end{itemize}
        \end{exercice}

        \begin{exercice}
            Calculer la valeur de chacune des expressions littérales suivantes pour $a = -2$ et $b = 3$.
            \begin{multicols}{2}
                \begin{itemize}
                    \item $A = 3ab - b$
                    \item $B = 2a + 5b$
                    \item $C = -5a + 2b$
                    \item $D = -2a - b$
                    \item $E = 3ab - a$
                    \item $F = 4b + 2ab$
                \end{itemize}
            \end{multicols}
        \end{exercice}
    \end{multicols}
\end{Maquette}

\end{document}
