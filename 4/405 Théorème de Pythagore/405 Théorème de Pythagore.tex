% !TEX program = lualatex
\documentclass[11pt]{article}

% -------- LuaLaTeX : polices et langue --------
\usepackage{fontspec}
\setmainfont{Latin Modern Roman}
\setsansfont{Tex Gyre Heros}
%\renewcommand{\familydefault}{\sfdefault} % force le sans serif par défaut
\usepackage{polyglossia}
\setdefaultlanguage{french}

% -------- Mise en page --------
\usepackage[a4paper,margin=1cm]{geometry}
\usepackage{multicol}
\usepackage{fancyhdr}
\pagestyle{empty}
\usepackage[most]{tcolorbox}

% -------- Mathématiques --------
\usepackage{amsmath,amssymb,mathtools}
\usepackage{icomma}
% \sisetup{locale=FR}

\usepackage{enumitem}
\setlist[itemize]{left=0pt}
\setlist[enumerate]{left=0pt, label=\textbf{\alph*}.}

\usepackage{ProfCollege}
\usepackage{ProfMaquette}

\usepackage{tabularray}

% -------- Divers --------
\newcommand{\ligne}{{\color{gray!60}\hrulefill}}
\setlength{\parindent}{0pt}

\begin{document}


\begin{Maquette}[Fiche]{Theme=Théorème de Pythagore, Niveau=Quatrième}

    \begin{multicols}{2}

        \begin{exercice}
            Dans chaque cas, calcule l’aire d’un carré …
            \begin{multicols}{2}
                \begin{enumerate}
                    \item de côté \Lg{6}
                    \item de côté \Lg[dm]{5}
                    \item de côté \Lg[mm]{11}
                    \item de côté \Lg{35}
                    \item de côté \Lg[m]{3,4}
                    \item de côté \Lg[m]{120}
                \end{enumerate}
            \end{multicols}
            Réalise à chaque fois un schéma et reporte les différentes valeurs dessus.
        \end{exercice}

        \begin{exercice}
            À l’aide de ta calculatrice, complète les égalités suivantes.

            Ne tape aucune multiplication sur ta calculatrice, mais utilise directement la touche Carré : $x ^2$, $\blacksquare ^2$ ou $\square ^2$.

            \begin{multicols}{2}
                \begin{enumerate}
                    \item $153^2 = \ldots \ldots \ldots$
                    \item $17,5^2 = \ldots \ldots \ldots$
                    \item $5,8^2  = \ldots \ldots \ldots$
                    \item $12,23^2  = \ldots \ldots \ldots$
                    \item $17,02^2 = \ldots \ldots \ldots$
                    \item $1006^2 = \ldots \ldots \ldots \ldots$
                \end{enumerate}
            \end{multicols}
        \end{exercice}


        \begin{exercice}
            Un carré a une aire de \Aire{81}. Quelle est la longueur de son côté ? Réalise un schéma et reporte les valeurs dessus.
        \end{exercice}



        \begin{exercice}
            À l’aide de ta calculatrice, complète les égalités suivantes.

            \begin{multicols}{2}
                \begin{enumerate}
                    \item $\sqrt{256} = \ldots \ldots$
                    \item $\sqrt{62001} = \ldots \ldots$
                    \item $\sqrt{289} = \ldots \ldots$
                    \item $\sqrt{7140,25} = \ldots \ldots$
                    \item $\sqrt{357,21} = \ldots \ldots$
                    \item $\sqrt{18,29} = \ldots \ldots$
                    \item $\sqrt{157} = \ldots \ldots$
                    \item $\sqrt{40} = \ldots \ldots$
                \end{enumerate}
            \end{multicols}
        \end{exercice}

        \begin{exercice}
            \includegraphics[width=\linewidth]{Images/ex4.png}
        \end{exercice}

        

    \end{multicols}

     \begin{exercice}
        On a construit des carrés sur les trois côtés de certains triangles. Dans chaque cas :
        \begin{itemize}
            \item calcule l’aire des trois carrés
            \item compare l’aire du plus grand carré avec l’aire des deux petits carrés réunis
        \end{itemize}

    \end{exercice}
 
\setlength{\columnsep}{1cm}
\begin{multicols}{2}
    \underline{Situation 1 :}
    \begin{center}
        \includegraphics[width=.8\linewidth]{Images/1.png}
    \end{center}
    Aire du grand carré : \ligne

    \vspace{.4cm}
    Aire des deux petits réunis : \ligne

    \columnbreak
    \underline{Situation 2 :}
    \begin{center}
        \includegraphics[width=.8\linewidth]{Images/2.png}
    \end{center}
    Aire du grand carré : \ligne

    \vspace{.4cm}
    Aire des deux petits réunis : \ligne

\end{multicols}
\newpage
\begin{multicols}{2}
    \underline{Situation 3 :}
    \begin{center}
        \includegraphics[width=.8\linewidth]{Images/4.png}
    \end{center}
    Aire du grand carré : \ligne

    \vspace{.4cm}
    Aire des deux petits réunis : \ligne


    \columnbreak
    \underline{Situation 4 :}
    \begin{center}
        \includegraphics[width=.8\linewidth]{Images/5.png}
    \end{center}
    Aire du grand carré : \ligne

    \vspace{.4cm}
    Aire des deux petits réunis : \ligne

\end{multicols}
\vspace{1cm}
\begin{multicols}{2}
    \underline{Situation 5 :}
    \begin{center}
        \includegraphics[width=.8\linewidth]{Images/3.png}
    \end{center}
    Aire du grand carré : \ligne

    \vspace{.4cm}
    Aire des deux petits réunis : \ligne

    \columnbreak
    \underline{Situation 6 :}
    \begin{center}
        \includegraphics[width=.8\linewidth]{Images/6.png}
    \end{center}
    Aire du grand carré : \ligne

    \vspace{.4cm}
    Aire des deux petits réunis : \ligne

\end{multicols}

    \newpage
    \begin{exercice}
        Dans chaque cas, calcule la longueur manquante, rédige proprement sur ton cahier.
        \begin{multicols}{2}
            \begin{center}
                \includegraphics[width=\linewidth]{Images/ex7-1.png}
            \end{center}

            \columnbreak
            \begin{center}
                \includegraphics[width=\linewidth]{Images/ex7-2.png}

            \end{center}
        \end{multicols}
    \end{exercice}

    \newpage
    \begin{exercice}
        Dans chaque cas, complète l’égalité de Pythagore.
        \begin{multicols}{2}
            \begin{center}
                \includegraphics[width=.8\linewidth]{Images/eg-01.png}
            
            $\textrm{GH}^2 \, \, = \, \, \ldots\ldots \, \, + \, \, \ldots\ldots$
            \end{center}
            \vspace{.3cm}

            \begin{center}
                \includegraphics[width=.8\linewidth]{Images/eg-02.png}

                $\textrm{WX}^2 \, \, = \, \, \ldots\ldots \, \, \ldots \, \, \ldots\ldots$
            \end{center}
            \vspace{.3cm}

            \begin{center}
                \includegraphics[width=.8\linewidth]{Images/eg-03.png}

                $\ldots\ldots \, \, = \, \, \textrm{RS}^2 \, \, + \, \, \ldots\ldots$
            \end{center}
            \vspace{.3cm}



            \begin{center}
                \includegraphics[width=.8\linewidth]{Images/eg-04.png}

                   $\ldots\ldots \, \, = \, \, \dots\ldots \, \, + \, \, \ldots\ldots$
                \end{center}
            \vspace{.3cm}

            \columnbreak

            \begin{center}
                \includegraphics[width=.8\linewidth]{Images/eg-05.png}
                
                $\textrm{MK}^2 \, \, = \, \, \ldots\ldots \, \, - \, \, \ldots\ldots$
            \end{center}
            \vspace{.3cm}




            \begin{center}
                \includegraphics[width=.8\linewidth]{Images/eg-06.png}

                $\textrm{EF}^2 \, \, = \, \, \ldots\ldots \, \, \ldots \, \, \ldots\ldots$
            \end{center}
            \vspace{.3cm}



            \begin{center}
                \includegraphics[width=.8\linewidth]{Images/eg-07.png}


                $\ldots\ldots \, \, = \, \, \ldots\ldots \, \, - \, \, \textrm{IJ}^2$
            \end{center}
            \vspace{.3cm}



        \end{multicols}
    \end{exercice}

    \begin{exercice}
        \begin{enumerate}
            \item Le triangle MNP est rectangle en M avec MN = \Lg{5,2} et MP = \Lg{4,8}. Calcule la valeur de NP arrondie au dixième d’unité.
            \item Calcule BC. Donne la valeur approchée au centième d’unité près.
            \begin{center}
                \includegraphics[width=.25\linewidth]{Images/ex9.png}
            \end{center}
        \end{enumerate}
    \end{exercice}

    \newpage

    \begin{exercice}
        Effectue sur ton cahier une rédaction détaillée des calculs des longueurs LA et TN.
        
        Donne à cahque fois la valeur exacte, et une valeur approchée au centième d’unité.

        \begin{center}
        \includegraphics[width=.55\linewidth]{Images/ex10.png}
        \end{center}
    \end{exercice}

    \begin{multicols}{2}
        \begin{exercice}
            Lors de son déménagement, Alix doit transporter son réfrigérateur dans un camion. Pour le mettre à l’intérieur, il le pose sur le bord comme indiqué sur la figure suivante (qui n’est pas à l’échelle).

            \begin{center}
            \includegraphics[width=\linewidth]{Images/ex11.png}
            \end{center}

            Alix pourra-t-il redresser le réfrigérateur en position verticale pour le rentrer dans le camion sans bouger le point d’appui A ?
        \end{exercice}

        \columnbreak

        \begin{exercice}
            Superman a raté son atterrissage !
            
            Il a percuté un arbre, qui est à présent … plié en deux.
            
            On sait que PH = \Lg[m]{5} et IH = \Lg[m]{7}.
            
            Calcule la hauteur de l’arbre avant que Superman ne le percute. 
            
            Arrondis au dixième d’unité près.

            \begin{center}
                \includegraphics[width=\linewidth]{Images/ex12.png}
            \end{center}
        
        \end{exercice}
    \end{multicols}
    \newpage
    \begin{exercice}
        Détaille sur ton cahier les calculs nécessaires afin de déterminer la longueur de l’hypoténuse du triangle L.
        \begin{center}
            \includegraphics[width = .65\linewidth]{Images/pilePythagore.png}
        \end{center}

    \end{exercice}
\end{Maquette}

\end{document}
