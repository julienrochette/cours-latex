% !TEX program = lualatex
\documentclass[11pt]{article}

% -------- LuaLaTeX : polices et langue --------
\usepackage{fontspec}
\setmainfont{Latin Modern Roman}
\setsansfont{Tex Gyre Heros}
%\renewcommand{\familydefault}{\sfdefault} % force le sans serif par défaut
\usepackage{polyglossia}
\setdefaultlanguage{french}

% -------- Mise en page --------
\usepackage[a4paper,margin=1cm]{geometry}
\usepackage{multicol}
\usepackage{fancyhdr}
\pagestyle{empty}
\usepackage[most]{tcolorbox}

% -------- Mathématiques --------
\usepackage{amsmath,amssymb,mathtools}
\usepackage{icomma}
% \sisetup{locale=FR}

\usepackage{enumitem}
\setlist[itemize]{left=0pt}
\setlist[enumerate]{left=0pt, label=\textbf{\alph*}.}

\usepackage{ProfCollege}
\usepackage{ProfMaquette}

\usepackage{tabularray}

% -------- Divers --------
\setlength{\parindent}{0pt}

\begin{document}


\begin{Maquette}[Fiche]{Theme=Théorème de Pythagore, Niveau=Quatrième}

    \begin{multicols}{2}
        
        \begin{exercice}
            Dans chaque cas, calcule l’aire d’un carré …
            \begin{multicols}{2}
                \begin{enumerate}
                    \item de côté \Lg{6}
                    \item de côté \Lg[dm]{5}
                    \item de côté \Lg[mm]{11}
                    \item de côté \Lg{35}
                    \item de côté \Lg[m]{3,4}
                    \item de côté \Lg[m]{120}
                \end{enumerate}
            \end{multicols}
            Réalise à chaque fois un schéma et reporte les différentes valeurs dessus.
        \end{exercice}

        \begin{exercice}
            À l’aide de ta calculatrice, complète les égalités suivantes. 
            
            Ne tape aucune multiplication sur ta calculatrice, mais utilise directement la touche Carré : $x ^2$, $\blacksquare ^2$ ou $\square ^2$.

            \begin{multicols}{2}
                \begin{enumerate}
                    \item $153^2 = \ldots \ldots \ldots$
                    \item $17,5^2 = \ldots \ldots \ldots$
                    \item $5,8^2  = \ldots \ldots \ldots$
                    \item $12,23^2  = \ldots \ldots \ldots$
                    \item $17,02^2 = \ldots \ldots \ldots$
                    \item $1006^2 = \ldots \ldots \ldots \ldots$
                \end{enumerate}
            \end{multicols}
        \end{exercice}

        
        \begin{exercice}
            Un carré a une aire de \Aire{81}. Quelle est la longueur de son côté ? Réalise un schéma et reporte les valeurs dessus.
        \end{exercice}


        
        \begin{exercice}
            À l’aide de ta calculatrice, complète les égalités suivantes. 
            
            \begin{multicols}{2}
                \begin{enumerate}
                    \item $\sqrt{256} = \ldots \ldots$
                    \item $\sqrt{62001} = \ldots \ldots$
                    \item $\sqrt{289} = \ldots \ldots$
                    \item $\sqrt{7140,25} = \ldots \ldots$
                    \item $\sqrt{357,21} = \ldots \ldots$
                    \item $\sqrt{18,29} = \ldots \ldots$
                    \item $\sqrt{157} = \ldots \ldots$
                    \item $\sqrt{40} = \ldots \ldots$
                \end{enumerate}
            \end{multicols}
        \end{exercice}

        \begin{exercice}
            \includegraphics[width=\linewidth]{Images/ex4.png}
        \end{exercice}

    \end{multicols}
\end{Maquette}

\end{document}
