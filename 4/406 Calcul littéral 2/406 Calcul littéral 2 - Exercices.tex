% !TEX program = lualatex
\documentclass[11pt]{article}

% -------- LuaLaTeX : polices et langue --------
\usepackage{fontspec}
\setmainfont{Latin Modern Roman}
\setsansfont{Tex Gyre Heros}
%\renewcommand{\familydefault}{\sfdefault} % force le sans serif par défaut
\usepackage{polyglossia}
\setdefaultlanguage{french}

% -------- Mise en page --------
\usepackage[a4paper,margin=1cm]{geometry}
\usepackage{multicol}
\usepackage{fancyhdr}
\pagestyle{empty}
\usepackage[most]{tcolorbox}

% -------- Mathématiques --------
\usepackage{amsmath,amssymb,mathtools}
\usepackage{icomma}
% \sisetup{locale=FR}

\usepackage{enumitem}
\setlist[itemize]{left=0pt}
\setlist[enumerate]{left=0pt, label=\textbf{\arabic*}.}

\usepackage{ProfCollege}
\usepackage{ProfMaquette}

\usepackage{tabularray}

% -------- Divers --------
\setlength{\parindent}{0pt}

\begin{document}

\begin{multicols}{2}

    \begin{Maquette}[Fiche]{Theme=Calcul littéral 2, Niveau=Quatrième}

        \begin{exercice}
            Pour calculer $4 \times 17$, Marie a écrit :

            \[ 4 \times 17 \]
            \[ 4 \times (10 + 7) \]
            \[ 4 \times 10 \, \, + \, \, 4 \times 7 \]
            \[ 40 + 28 \]
            \[ 68 \]

            \begin{enumerate}
                \item Utilise le même principe pour calculer :
                      \begin{itemize}
                          \item $6 \times 13$
                          \item $5 \times 31$
                          \item $8 \times 23$
                          \item $53 \times 6$
                      \end{itemize}
                \item Comment s’appelle la propriété de l’opération $\times$ que Marie a utilisée lors du passage de sa deuxième à sa troisième ligne ?
                \item Utilise cette propriété pour réduire les expressions littérales suivantes :
                      \begin{itemize}
                          \item $\textrm{A} = 5 \times (x + 7)$
                          \item $\textrm{B} = 2 (4y + 5)$
                          \item $\textrm{C} = 9 \times ( 2 - a )$
                          \item $\textrm{D} = x \times (x + 7)$
                          \item $\textrm{E} = 2x (5 - 4x)$
                      \end{itemize}
            \end{enumerate}
        \end{exercice}
        \columnbreak
        \begin{exercice}
            Recopier, développer et réduire.
            \begin{multicols}{2}
                \textbf{Série 1}
                \begin{enumerate}[label=\textbf{\alph*.}]
                    \item $ 3 \times \left( a+2 \right) $
                    \item $ 4 \left( 3+x \right) $
                    \item $ s \times \left( s+2 \right) $
                    \item $ 3u \left( u+4 \right) $
                    \item $ 5 \left( 4+a \right) $
                    \item $ s \left( 3+4s \right) $
                \end{enumerate}
                \textbf{Série 2}
                \begin{enumerate}[label=\textbf{\alph*.}]
                    \item $ 3 \times \left( 2+3a \right) $
                    \item $ 2s \left( 2+4s \right) $
                    \item $ \left( x+3 \right) \times 2 $
                    \item $ 5 \left( 2+x+y \right) $
                    \item $ x^2 \times \left( 3x+5 \right) $
                    \item $ a^2 \left( 3+a \right) $
                \end{enumerate}
            \end{multicols}
        \end{exercice}

        \begin{exercice}
            Recopier, développer et réduire.
            \begin{multicols}{2}
                \textbf{Série 1}
                \begin{enumerate}[label=\textbf{\alph*.}]
                    \item $ 2 \times \left( x-3 \right) $
                    \item $ 3 \times \left( 2s-1 \right) $
                    \item $ 2m \left( 5-4m \right) $
                    \item $ 5x \times \left( -2x+3 \right) $
                    \item $ 3 \times \left( -2-a \right) $
                    \item $ a \times \left( 2-2a \right) $
                \end{enumerate}
                \textbf{Série 2}
                \begin{enumerate}[label=\textbf{\alph*.}]
                    \item $ 2 \times \left( 3a+3 \right) $
                    \item $ 2x \times \left( x-2 \right) $
                    \item $ -4 \times \left( -3-x \right) $
                    \item $ -5x \times \left( 4x+3 \right) $
                    \item $ 5 \times \left( -4+4m \right) $
                    \item $ -a \times \left( -4a+2 \right) $
                \end{enumerate}
            \end{multicols}
        \end{exercice}

        \newpage

        \begin{exercice}
            On considère le programme de calcul suivant :

            \begin{center}
                \includegraphics[width=\linewidth]{images/programmeCalcul.png}
            \end{center}

            \begin{enumerate}
                \item Utiliser le programme de calcul avec 8 comme nombre de départ. Recommencer avec 11.
                \item On appelle $x$ le nombre de départ. Exprimer le résultat final en fonction de $x$.
                \item Développer et réduire l’expression obtenue. Qu’en conclure ?
            \end{enumerate}
        \end{exercice}
        \columnbreak

        \begin{exercice}
            Voici un nouveau programme de calcul, qu’on peut appliquer avec n’importe quel nombre inférieur à 1000 :

            \begin{center}
                \includegraphics[width=\linewidth]{images/programmeCalcul2.png}
            \end{center}

            \begin{enumerate}
                \item À l’aide de la calculatrice, utiliser le programme de calcul avec 123 comme nombre de départ. Recommencer avec 457. Puis avec 628.
                \item Quelle \emph{conjecture} peut-on faire ?
                \item On note $x$ le nombre de départ. Exprimer le résultat du programme de calcul en fonction de $x$. Réduire cette expression et conclure.
            \end{enumerate}
        \end{exercice}
    \end{Maquette}

\end{multicols}

\end{document}
