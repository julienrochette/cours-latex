% !TEX program = lualatex
\documentclass[11pt]{article}

% -------- LuaLaTeX : polices et langue --------
\usepackage{fontspec}
\setmainfont{Latin Modern Roman}
\setsansfont{Tex Gyre Heros}
%\renewcommand{\familydefault}{\sfdefault} % force le sans serif par défaut
\usepackage{polyglossia}
\setdefaultlanguage{french}

% -------- Mise en page --------
\usepackage[a4paper,margin=1cm]{geometry}
\usepackage{multicol}
\usepackage{fancyhdr}
\pagestyle{empty}
\usepackage[most]{tcolorbox}

% -------- Mathématiques --------
\usepackage{amsmath,amssymb,mathtools}
\usepackage{icomma}
% \sisetup{locale=FR}

\usepackage{enumitem}
\setlist[itemize]{left=0pt}
\setlist[enumerate]{left=0pt, label=\textbf{\alph*}.}

\usepackage{ProfCollege}
\usepackage{ProfMaquette}

\usepackage{tabularray}

% -------- Divers --------
\setlength{\parindent}{0pt}

\newcommand{\ligne}{{\color{gray!60}\hrulefill}}

\begin{document}


\begin{Maquette}[Fiche]{Theme=Théorème de Pythagore, Niveau=Quatrième}

    \begin{multicols}{5}
        \begin{exercice}\end{exercice}
        \begin{exercice}\end{exercice}
        \begin{exercice}\end{exercice}
        \begin{exercice}\end{exercice}
        \begin{exercice}\end{exercice}
    \end{multicols}
    \vspace{1cm}
    \begin{exercice}
        On a construit des carrés sur les trois côtés de certains triangles. Dans chaque cas :
        \begin{itemize}
            \item calcule l’aire des trois carrés
            \item compare l’aire du plus grand carré avec l’aire des deux petits carrés réunis
        \end{itemize}

    \end{exercice}
\end{Maquette}
\setlength{\columnsep}{1cm}
\begin{multicols}{2}
    \underline{Situation 1 :}
    \begin{center}
        \includegraphics[width=.8\linewidth]{Images/1.png}
    \end{center}
    Aire du grand carré : \ligne

    \vspace{.4cm}
    Aire des deux petits réunis : \ligne

    \columnbreak
    \underline{Situation 2 :}
    \begin{center}
        \includegraphics[width=.8\linewidth]{Images/2.png}
    \end{center}
    Aire du grand carré : \ligne

    \vspace{.4cm}
    Aire des deux petits réunis : \ligne

\end{multicols}
\newpage
\begin{multicols}{2}
    \underline{Situation 3 :}
    \begin{center}
        \includegraphics[width=.8\linewidth]{Images/4.png}
    \end{center}
    Aire du grand carré : \ligne

    \vspace{.4cm}
    Aire des deux petits réunis : \ligne


    \columnbreak
    \underline{Situation 4 :}
    \begin{center}
        \includegraphics[width=.8\linewidth]{Images/5.png}
    \end{center}
    Aire du grand carré : \ligne

    \vspace{.4cm}
    Aire des deux petits réunis : \ligne

\end{multicols}
\vspace{1cm}
\begin{multicols}{2}
    \underline{Situation 5 :}
    \begin{center}
        \includegraphics[width=.8\linewidth]{Images/3.png}
    \end{center}
    Aire du grand carré : \ligne

    \vspace{.4cm}
    Aire des deux petits réunis : \ligne

    \columnbreak
    \underline{Situation 6 :}
    \begin{center}
        \includegraphics[width=.8\linewidth]{Images/6.png}
    \end{center}
    Aire du grand carré : \ligne

    \vspace{.4cm}
    Aire des deux petits réunis : \ligne

\end{multicols}

\end{document}
