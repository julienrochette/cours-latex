% !TEX program = lualatex
\documentclass[10pt,a5paper]{article}

% -------- LuaLaTeX : polices et langue --------
\usepackage{fontspec}
\setmainfont{Latin Modern Roman}
\setsansfont{Tex Gyre Heros}
\renewcommand{\familydefault}{\sfdefault} % force le sans serif par défaut
\usepackage{polyglossia}
\setdefaultlanguage{french}

% -------- Mise en page --------
\usepackage[a5paper,margin=2cm]{geometry}
\usepackage{multicol}
\usepackage{fancyhdr}
\pagestyle{empty}
\usepackage[most]{tcolorbox}

% -------- Mathématiques --------
\usepackage{amsmath,amssymb,mathtools}
% \usepackage{siunitx}
% \sisetup{locale=FR}

\usepackage{enumitem}
\usepackage{ProfCollege}

% -------- Divers --------
\setlength{\parindent}{0pt}

\begin{document}
\PfCPanneaux[FeuTricolore]
\begin{tcolorbox}[colback=teal!10!white, colframe=teal!80!black]
\begin{center}
\large\textbf{Tâches de rentrée}
\end{center}
\end{tcolorbox}

\vspace{0.6em}
Le plus rapidement possible :

\begin{itemize}[leftmargin=1.5em, label=$\square$]
    \item compléter et rendre le coupon \textbf{Autorisations entrée/sortie}
\end{itemize}

\vspace{0.6em}
Pour le vendredi 12 septembre au plus tard :
\begin{itemize}[leftmargin=1.5em, label=$\square$]
\item  compléter et rendre le coupon \textbf{Réunion parents lundi 15/9}
\item  compléter les 2\textsuperscript{e} et 4\textsuperscript{e} de couverture du \textbf{carnet de liaison}, coller une \textbf{photo}
\item  lire et signer, élève et parent(s), le \textbf{règlement intérieur} dans le carnet de liaison
\item  compléter et rendre le coupon \textbf{Médicaments}
\item  faire signer et rendre la fiche \textbf{État des manuels scolaires}
\item si ça n’a pas déjà été fait, transmettre au collège une \textbf{attestation d’assurance scolaire}
\end{itemize}
\vspace{1em}
Signature(s):

\end{document}
