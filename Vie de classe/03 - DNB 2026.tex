\documentclass[11pt,a5paper]{article}

% -------- LuaLaTeX : polices et langue --------
\usepackage{fontspec}
\setmainfont{Latin Modern Roman}
\setsansfont{Tex Gyre Heros}
\renewcommand{\familydefault}{\sfdefault} % force le sans serif par défaut
\usepackage{polyglossia}
\setdefaultlanguage{french}

% -------- Mise en page --------
\usepackage[a5paper, margin=1cm]{geometry}
\usepackage{fancyhdr}
\pagestyle{empty}
\setlength{\parindent}{0pt}
\usepackage[most]{tcolorbox}

\usepackage{lipsum}
\usepackage{enumitem}
\usepackage{hyperref}
% -------- Mathématiques --------
\usepackage{amsmath,amssymb,mathtools}

\begin{document}
\vspace*{2em}
\begin{tcolorbox}[colback=red!10!white, colframe=red!80!black]
\begin{center}
\Large\textbf{Nouveau mode de calcul pour le DNB 2026}
\end{center}
\end{tcolorbox}
\vspace{1em}
\begin{itemize}[leftmargin=1.1em,itemsep=1.2em]
\item \textbf{Contrôle continu} : moyenne des \emph{moyennes annuelles durant l’année de troisième} des 12 matières obligatoires\\
\vspace{0.2em}
(les points au dessus de 10 de l'option s’ajoutent \emph{avant} la division par 12)
\item \textbf{Épreuves terminales} : moyenne avec coefficients 
\subitem Français (2)
\subitem Mathématiques (2)
\subitem Histoire‑Géographie (1.5)
\subitem EMC (0.5)
\subitem Sciences (2)
\subitem Oral (2)
\item \textbf{Note finale} :
\vspace{-0.4em}
\[ \mathsf{\textsf{contrôle continu} \times 0,40 \,\, + \,\, \textsf{épreuves terminales} \times 0,60} \]
\item \textbf{Mentions} :
\subitem $\mathsf{[12;14[}\,\,$ Assez Bien
\subitem $\mathsf{[14; 16[}\,\,$ Bien
\subitem $\mathsf{[16; 18[}\,\,$ Très bien
\subitem $\mathsf{[18; 20]}\,\,$ Très bien les avec félicitations du jury
\end{itemize}

\section*{Calculateur en ligne}
\url{https://calcul-points-brevet.fr/}
\newpage

\section*{Contrôle continu de l’année de 3\textsuperscript{e}}
\begin{table}[h!]
{\centering
\renewcommand{\arraystretch}{1.5} 
\begin{tabular}{l c c c c}
\hline
Matière & Trimestre 1 & Trimestre 2 & Trimestre 3 & Année \\
\hline
Français & 17 & 16 & 16 & 16,33 \\
Mathématiques & 15 & 14 & 13 & 14 \\
Histoire-Géographie & 9 & 12 & 10 & 10,33 \\
E.M.C. & 13 & 13 & 15 & 13,67 \\
Physique-Chimie & 16 & 13 & 14 & 14,33 \\
S.V.T. & 9 & 8 & 10 & 9 \\
Technologie & 15 & 11 & 10 & 12 \\
Langue vivante 1 & 14 & 14 & 17 & 15 \\
Langue vivante 2 & 13 & 17 & 16 & 15,33 \\
Arts plastiques & 11 & 9 & 13 & 11 \\
Éducation musicale & 14 & 15 & 11 & 13,33 \\
E.P.S. & 19 & 15 & 16 & 16,67 \\
\hline
\end{tabular}
}
\end{table}

\vspace{1em}
Une moyenne annuelle en Chant choral de 16/20 donne 6 points bonus.

\vspace{1.3em}
Le résultat au contrôle continu est donc :
\[
	\mathsf{
		\dfrac{16,33+14+\ldots+13,33+16,67+6}{12} \,\, \simeq \,\, 13,92
	}
\]


\newpage
\section*{Épreuves terminales du DNB}
\begin{table}[h!]
{\centering
\renewcommand{\arraystretch}{1.5} 
\begin{tabular}{l c c}
\hline
Épreuve & Coefficient & Note /20\\
\hline
Français & 2 & 14 \\
Mathématiques & 2 & 17 \\
Histoire-Géographie & 1,5 & 15 \\
E.M.C. & 0,5 & 13 \\
Sciences & 2 & 14 \\
Oral & 2 & 17 \\
\hline
\end{tabular}
}
\end{table}

Le résultat aux épreuves terminales est :
\[\mathsf{
	\dfrac{
		2 \times 14 \, + \, 2 \times 17 \, + \, 1,5 \times 15 \, + \, 0,5 \times 13 \, + \, 2 \times 14 \, + \, 2 \times 17
	}{10} \, \simeq \, 15,3
}\]

\vspace{1.3em}
\section*{Note finale au DNB}
\vspace{-1em}
\[ \mathsf{13,92 \times 0,40 \,\, + \,\, 15,3 \times 0,60 \,\, \simeq 14,75 }\]
\vspace{0,4em}
DNB obtenu avec mention Bien car $\mathsf{14 \leq 14,75 < 16}$.
\end{document}
